\subsection{Use Cases}
\begin{figure}[htb]
	\centering
	\includegraphics[width=0.9\textwidth]{reqspec/UseCase.png}
	\caption{Use case}
	\label{fig:usecase}
\end{figure}

There is only one actor, the generic user

\begin{tabular}[t]{|l|p{0.8\textwidth}|}\hline
Use Case&1 - Create New Observation\\\hline
Preconditions&User wants to register an observation\\\hline
Flow&1. User taps the new observation button\newline
2. User selects location from list of close locations, or selects gps location\newline
3. User adds one species. Helped by autocomplete\newline
4. User saves the observation.\\\hline
Extensions& 1a. Add number observed to species\newline
3a. Add more info to the species, see Use Case 2\newline
3b. Add more species into observation, see Use Case 3\\\hline
Postcondtions&A new observation has been saved and the user is directed back to the main menu.\\\hline
Complexity&Medium\\\hline
Priority&High\\\hline
\end{tabular}

\hspace{2em}

\begin{tabular}[t]{|l|p{0.8\textwidth}|}\hline
Use Case&2 - Add More Information to Species\\\hline
Preconditions&User wants to specify more details about an observation\\\hline
Flow& 1. User taps the the species row to bring up the new detailwindow.\newline
2. User selects addtional info to enter from such categories as \newline
Activity (drop down?) \newline
Age\newline
Sex\newline
Start Date\newline
Start Time\newline
End date \newline
End Time \newline
Comment \newline
Picture \newline
%	\begin{itemize}
%		\item Activity (drop down?)
%		\item age
%		\item sex
%		\item date
%		\item time
%		\item enddate
%		\item endtime
%		\item commen
%		\item picture
%\end{itemize}
3. User taps ok to get back to the main observation window.\\\hline
Extensions& \\\hline
Postcondtions&Additonal information about an observation has been saved.\\\hline
Complexity&Low\\\hline
Priority&High\\\hline
\end{tabular}

\hspace{2em}

\begin{tabular}[t]{|l|p{0.8\textwidth}|}\hline
Use Case&3 - Add another species to observation\\\hline
Preconditions&User wants to add another species to the observation\\\hline
Flow&1. User taps 'Add species' button.\newline
2. Optionally selects another location, otherwise the same one is selected.
3. User selects species helped by auto complete.
4. User taps 'OK' to go back to main observation window \\\hline  
Extensions& \\\hline
Postcondtions&Additonal species has been added to the observation\\\hline
Complexity&Low\\\hline
Priority&High\\\hline
\end{tabular}

\hspace{2em}

\begin{tabular}[t]{|l|p{0.8\textwidth}|}\hline
Use Case&4  Export  observations\\\hline
Preconditions& User wants to export their observations \\\hline
Flow&1. User taps the 'Export' button on the main screen
2. User selects the observations to be exported
3. User taps the export button\\\hline
Extensions& \\\hline
Postcondtions&Observations are exported in excel format to the user's email so they can be imported into the online system later\\\hline
Complexity&Low\\\hline
Priority&High\\\hline
\end{tabular}

\hspace{2em}

\begin{tabular}[t]{|l|p{0.8\textwidth}|}\hline
Use Case&5 Take picture\\\hline
Preconditions&User wants to take a picture to be attached to an observation
The device has a camera\\\hline
Flow&1. User taps 'Take picture' button.
2. User takes picture \\\hline
Extensions& 2a. User adds a description to the picture\\\hline
Postcondtions&PIcture is stored on the phone with an easily recognizable filename so it can be attached to an observation later\\\hline
Complexity&Low\\\hline
Priority&Medium\\\hline
\end{tabular}

\hspace{2em}

\begin{tabular}[t]{|l|p{0.8\textwidth}|}\hline
Use Case&6 - View observations\\\hline
Preconditions&User wants to view locally stored observations\\\hline
Flow&1. User taps the 'View observations' button.
2. User selects the observation from a list of stored observations \\\hline
Extensions& \\\hline
Postcondtions&\\\hline
Complexity&Low\\\hline
Priority&High\\\hline
\end{tabular}