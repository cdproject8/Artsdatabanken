\subsection{Functional Requirements}
The functional requirements are listed in this section. They are meant as a short descirption of each requirement, and are further elaborated by the use cases.

\subsubsection{Priority}
The priorities are set according to the following description.
\begin{itemize}
	 \item H(High) - These requirements are essential for the prototype to work satisfactory. These requirements will be prioritized first.
	\item M(Medium) - These requirements are important for the prototype to work satisfacory. They are important, but not critical for the prototype.
	\item L(Low) - These requirements are not important for the prototype. They are "nice to have", but will be implemented last.
\end{itemize}

\begin{table}
	\begin{tabular}[t]{|l|p{0.5\textwidth}|l|l|p{0.15\textwidth}|}\hline
	\bf ID&\bf Requirement& \bf Priority& \bf Complexity&\bf Time estimate\\\hline
	F1&User must be able to create a new observation &High&Medium&0\\\hline
	F2&User must be able to add more information to a species added to the observation (see use case for details) 	&High&Low&0\\\hline
	F3&User must be able to add more species to an observation &High&Low&0\\\hline
	F4&User must be able to export stored observations so they can be uploaded to Artsdatabanken later &High&High&0\\\hline
	F5&User must be able to take a picture with the device's camera &Medium&Medium&0\\\hline
	F6&User must be able to view observations tha are stored on the device &Low&Medium&0\\\hline
	F7&User must be able to edit observations that are stored on the device &Low&Low&0\\\hline
	F8&User must be able to update the local database of species and locations &Low&Medium&0\\\hline
	\end{tabular}
	\caption{Functional requirements}
	\label{funcreqs}
\end{table}