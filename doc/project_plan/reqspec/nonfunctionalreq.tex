\subsection{Nonfunctional Requirements}
Non-functional requirements are listed in this section. They are meant as a short description of each requirement that is further elaborated in the use cases. Non-functional requirements are meant to specify the criteria used to judge the operation of a system, rather than specific behaviors.

\subsubsection{User interface}
	This application is going to be used on the mobile phone so a couple of requirements follow.
	\begin{itemize}
		\item The screen size can vary from device to device, the UI must therefore be "universal" or exist in several versions. 
		\item The buttons/fields must be simple to use in different screen sizes.
	\end{itemize}
\subsubsection{Performance Requirements}
	To make this application succeed, the performance is very important. The things that this application will do, can also be done without the use of this application. By taking notes in a notebook and entering them in the database online from a computer. We will have to beat this in order for it to be useful.
	\begin{itemize}
		\item Must be faster than "take notes and enter them on a computer".
		\item Needs to be easy to use, and give the user the feeling that the application is very simple.
		\item The software needs to be fast, and have good performance without any loading times (if possible).
	\end{itemize}

\subsubsection{ Safety- and Security Requirements}
	All the data entered in this application  is available for everyone on the internet, and it is therefore not necessary to protect the data ntered. No specific safety or security requirements are known at this time.
	\begin{itemize}
		\item None.
	\end{itemize}
