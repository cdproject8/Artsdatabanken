\subsection{Security analysis and threat modeling}
In an environment where operation involves some inherent security risk, taking time to consider the security risks ahead of an actual crisis is well worth the effort. Having a web application can give one a greater audience, accessibility and scalable information dissemination platforms but at the same time it open wide a back door to your IT operations resources. Rogue code can create from minor inconvenience to completely crippling the system, causing a disruption in services. For web applications, it is incredibly hard to imagine where the security threat might come from or understand the motives of an attacker. Students trying to test their skill, IT vandalism or anyone looking for insight in to the internal operations of an organization can lurk in and wreak serious havoc unexpectedly.
\\[0.5cm]
There are two notable methods of security threat modeling techniques, Misuse case and Attack tree. Considering the nature of our project and the nature of content this application is going to deal with, and small sensitivity of the documents in the system, we opt to model our general security threats using Misuse case.
\begin{figure}[htb]
	\centering
    \includegraphics[width=0.9\textwidth]{reqspec/misusecase.png}
	\caption{Security threat model with Misuse case.}
	\label{fig:misusecase}
\end{figure}

\begin{table}[htb]
	\centering
    \begin{tabular}{| l | l |}
		\hline
		Potential threats & Mitigation suggestions \\
		\hline \hline
		Denial of service	& Thorough input validation \\& Clean exception handling \\
		Steal credentials	& Thorough input validation\\	& Strong authentication mechanism \\
        Deface content page & Thorough input validation\\
        Track observer location & Thorough input validation\\ & Clean picture metadata\\ & Secure communication path\\
        Impersonate & Strong authentication\\
        Push irrelevant information & Thorough input validation\\
        Deliberate record destruction & Thorough input validation\\
        
		\hline
    \end{tabular}
  \caption{Threat mitigation suggestions}
\end{table}
