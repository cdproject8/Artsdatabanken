\subsection{General Overview}

\subsubsection{ The process}
	In our first meeting with the customer we got information about what they do and what they want from us. Both functional and non-functional requirements were discussed in general. So our group had an overview of what we were going to do. We had some issues to take up with them about a non-functional requirement; this is described in detail later this section. After this meeting the group members made preliminary study to get a better overview of the task.
\indent Our task became more pinpointed in the second customer meeting. This time the meeting was in their premises, where we made a field-study and saw how the application was going to be used. Here we got more details on the functional requirements. After the field study we had a meeting with the customer, and discussed about the requirements (functional and non-functional). We talked about which functions that were most important, and made a priority of them. Some functions were more important to have than others.

\subsubsection{Purpose of the project}

The purpose of this project is to deliver an application that will satisfy the customers expectations.  
This application is meant to be a facilitator for the users of Artsdatabanken, that will make the registration of an observation simpler and more effective. Now the observator have to make notes (in a book) while observing, and then enter the noted data on Artsdatabanken's web page. The applications purpose is to replace the notebook, so users can make notes in the application. This way the observator does not need to use time to enter the observation data on the web page manually. That will be done by the mobile application. Saved data on the mobile will be exported to a format that is easily submittable on the website, when the mobile has access to an internet connection.

To reach this purpose, the application needs to be simple to use and have sufficient functions to be preferred over a notebook. 



\subsubsection{Target Audience}
	The Application is targeted to both professional users who will "use whatever they're given", yet have different main focuses than more casual users.

	The (main) target of this application will be a group that already are familiar with Artsdatabanken. In general they will have enough knowledge of registering data on the webpage, and will therefore not have difficulties with the use of this application. 

	Also the possible new user groups who might start using the services of Artsdatabanken if given an easy to use interface.


\subsubsection{Project Scope}
The application will be used to enter data of observations done. One observation will have sufficient data fields necessary. And it will be possible to make multiple observations. Stored observations in the mobile can be exported to the web page. Earlier made observations may also be edited and re-exported.

	This projects scope is to deliver the correct data to the web page. The further process is not covered in this project.
