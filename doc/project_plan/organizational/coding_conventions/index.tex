\subsection{Code conventions}

\subsubsection{JavaScript} \hspace{1mm}

JavaScript code should never be embedded in HTML files. Script tags should be
placed as late in the body as possible. \cite{crockford:code}

\paragraph{Line length and indentation} \hspace{1mm}

Use tabs for indentation. 

Lines should not be longer than 80 characters.

\paragraph{Comments and variable names} \hspace{1mm}

Keep comments at a minimum, they are disruptive to the code and
usually outdated. Write good code instead. Comments can be used to
clarify code that is not self-explanatory or to warn about issues like
framework bugs or weak code. Keep in mind that the code should already
be documented by tests you wrote before you implemented the feature.

Variables should have descriptive names. Try to find good names before
resorting to comments.

To indicate that a variable refers to jQuery objects people sometimes
prefix it with \$, do not use this convention, it makes it hard to
spot actual jQuery calls.

\paragraph{Declarations} \hspace{1mm}

Variables should always be declared, this will prevent conflicts with globals.

Functions should be declared as follows:

\begin{lstlisting}
	function somename(...) {
		...
	}

	var somename = function(...) {
		...
	}
\end{lstlisting}

Never place a space between the function name and the first parenthesis.

\paragraph{Statements} \hspace{1mm}

If statements should have the following form (notice that the else statement is
not on the same line as the closing curly brace):

\begin{lstlisting}
	if (condition) {
		statements
	}
	else {
		statements
	}
\end{lstlisting}

A switch statement should have the following form:

\begin{lstlisting}
	switch (expression) {
		case condition:
			statements
			break;
		... more case ...
		default:
			statements
	}
\end{lstlisting}
