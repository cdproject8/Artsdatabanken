\subsection{Code conventions}

\subsubsection{JavaScript}

JavaScript code should never be embedded in HTML files. Script tags should be
placed as late in the body as possible. \cite{crockford:code}

\paragraph{Line length and indentation}

Use tabs for indentation. 

Lines should not be longer than 80 characters.

\paragraph{Comments and variable names}

Keep comments at a minimum, they are disruptive to the code and
usually outdated. Write good code instead.

Variables should have descriptive names. If you feel the need to add comments
it is an indication that you broke the preceding rule.

To indicate that a variable refers to jQuery objects people sometimes prefix it
with \$, do not use this convention, it makes it hard to spot actual jQuery
calls.

\paragraph{Declarations}

Variables should always be declared, this will prevent conflicts with globals.

Functions should be decleared as follows:

\begin{lstlisting}
	function somename(...) {
		...
	}

	var somename = function(...) {
		...
	}
\end{lstlisting}

Never place a space between the function name and the first paranthesis.

\paragraph{Statements}

If statements should have the following form (notice that the else statement is
not on the same line as the closing curly brace):

\begin{lstlisting}
	if (condition) {
		statements
	}
	else {
		statements
	}
\end{lstlisting}

A switch statement should have the following form:

\begin{lstlisting}
	switch (expression) {
		case condition:
			statements
			break;
		... more case ...
		default:
			statements
	}
\end{lstlisting}
