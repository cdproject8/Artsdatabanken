\subsection{Project description (TODO)}
The project is part of the course TDT4290 Customer Driven Project. This course is mandatory for all computer science majors, and its goal is to give the students experience with customer relations, project management and group dynamics in a real project.

\subsubsection{The customer}
Our customer is Artsdatabanken. Artsdatabanken is a company in the
Norwegian Biodiversity Information Center (NBIC) body that provides the
public with information on Norwegian species and ecosystems.
Artsdatabanken claims to have approximately 6 million species of plants,
insects, birds, large predators and of which 85\% are birds. 
% could ..and of which .. be moved to the next sentence?
Artsdatabanken keeps a series of database resources such as Red List,
Alien Species, Species name and Habitat databases together with Species
Map and Species Observations. Artsdatabanken depends on these projects
and databases to fully conduct its operations and
responsibilities.\cite{artsdatabanken:about} 

\subsubsection{People involved in the project}
The student group consists of seven fourth year students partaking the Computer Science program at the Norwegian University of Science and Technology. Two of these students are enrolled in the International student exhange program for Computer Science. These are in alphabetical order: Anders Søbstad Rye, Andreas Berg Skomedal, Dag-Inge Aas, Muhsin Günaydin, Nikola Djoric, Stian Liknes and Yonathan Redda. The student group is advised by PhD Candidate, Muhammad Asif, at IDI.

The customers representatives are Askild Olsen and Helge Sandmark. Askild Olsen will function as the product owner, and has the final word when decisions are made about the project. They are employees of Artsdatabanken.
% Maybe we could include some background information about Askild and Helge and
% their motivations for the project.

\subsubsection{Project drivers}
Artsdatabanken has a very skilled user base. Some of the users have for a long
time requested a mobile application for gathering observation data in the
field. This application would replace the old notebook method of gathering
information, automate collection of some data such as GPS coordinates, and
decrease the complexity involved in making observations of species, so that
beginners would have an easier time learning the process. In addition,
Artsdatabanken wants the application to help increase the number of
observations by increasing the number of users and significantly lessen the
complexity involved in registering observations.

\subsubsection{Problem domain}
In the current situation, an observer has to make the observation, write down
the number of individuals and alternatively take a picture. After this is done, % Number of individuals
% Number of individual whats, people? (/me is being difficult)
% Is taking a picture an alternative to counting the individuals?
the observer must go home, post everything to an online form and
alternatively upload pictures separately. Because of the complexity and the
lack of automation involved in this process, the smaller observations are often
overlooked, and doesn't get registered. Artsdatabanken wants as many
observations done as possible.

In addition, considering that Artsdatabanken is a public institution brings
additional demands to the project. Artsdatabanken needs to support as many
mobile devices as possible. This means that the application must work on
a variety of devices with different operating systems and screen sizes.

\subsubsection{Proposed solution}
Our proposed solution involves using a cross-compiling mobile application
framework, named Phonegap, that can deploy the same code on multiple platforms.
This will enable us to use the same codebase, but deploy on six different
platforms. The applications itself will automate a lot of the data collection
involved with making observations, and exports the data to a computer-readable
format suitable for parsing by Artsdatabankens systems. This solution will
lower the barrier for making and registering observations, enabling more users
to partake.

\subsubsection{Project objective}
The student group are expected to deliver a common mobile application codebase
that can be deployed on many different platforms in addition to a working
application demo that can export observations ready for parsing by the server.
The server-side APIs will be provided by Artsdatabanken. In addition, the
student group will deliver all research made into the problem domain to the
customer, including research into cross-platform frameworks and deployment.

\subsubsection{Available resources}
Artsdatabanken will provide user testing of the application. They will also
course the student group in taxonomy and hold a field study, giving the
students the opportunity to learn about observations and species.

The students will supply testing hardware for the Android platform, and an iPad
for testing iOS applications. This gives the students the opportunity to test
the application on the two most popular platforms in various screen sizes.

\subsubsection{Limitations TODO}
The project is scheduled to last from the 30th of August to the 24th of
November. The budget working hours per group member is 350 hours, giving the
total working hours available of 2450 person hours. This is equal to over one
year worth of person hours. 

As the students have no testing hardware for Symbian, Bada, BlackBerry or
Windows Phone, the applications developed for these platforms will not be
tested on a real device prior to launch. The students also lack hardware
necessary for efficient testing of the iOS platform, having to rely on sporadic
access to the hardware.
