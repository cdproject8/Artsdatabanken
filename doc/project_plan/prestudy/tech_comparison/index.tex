\subsection{Tech comparison}

\subsubsection{Android}

	Android is an operating system developed for a variety of mobile devices.
	It is based on the linux kernel, and provides a developer-friendly
	framework for app development (subset of Java, Dalvik virtual machine).
	Most of it's code is released under the Apache Licence, a free software
	licence.

	Android was listed as the best-selling smartphone platform worldwide in Q4
	2010 by Canalys. \cite{wiki:android}

	\paragraph{Pros}
		\begin{itemize}
			\item Can be developed using any major operating system (linux, windows, ...)
			\item Android devices can be emulated (AVD). This allows for testing on
			different screen sizes, etc.
			\item Developer-friendly
			\item Open source
		\end{itemize}

	\paragraph{Cons}
		\begin{itemize}
			\item Apps developed for Android will not run on iOS or Windows phones
		\end{itemize}

\subsubsection{iOS}

	iOS is Apple's mobile operating system.

	\paragraphs{Cons}
		\begin{itemize}
			\item Fashion product, lacks functionality
		\end{itemize}

	\begin{itemize}
		\item iOs development requires
			\begin{itemize}
				\item Macintosh running Mac OSX
				\item iFamily device for testing
			\end{itemize}
		\item iOS based applications can be developed on iPad or IPod Touch, but
		these devices lack accelerometer, compass, build-in GPS and camera
		accessibility facilities
		\item Apple is not fond of GPL or free software, this makes it
		problematic to use and develop open source solutions 
	\end{itemize}

\subsubsection{Phonegap}
PhoneGap is an HTML5 app platform that allows you to author native applications with web technologies and get access to APIs and app stores. PhoneGap leverages web technologies such as HTML and JavaScript. This makes you able to include PhoneGap is the only app platform available that can publish to 6 platforms. 
\cite{phonegap:about}
	\begin{itemize}
		\item Applications can be developed for Apple’s iOS, Google’s Android,
		Microsoft’s Windows Mobile, Nokia’s Symbian OS, RIM’s BlackBerry and Bada.
		\item Enables developers to take advantage of JavaScript, HTML5 and CSS3,
		which they might have already been familiar with.
		\item Access Native features such as compass, camera, network, media,
		notifications, sound, vibrate and storage etc.
    \item Can use existing CSS and Javascript libraries directly in your code
    \item Feels like a native application when in reality it's an offline web application
    \item Provides a build tool for automatically building binary application packages for six different platforms
    \item Provided under the (new) BSD license or alternativley the MIT licence, the framework is entierly Open Source and free for Open Source projects.
    \item Provides a well-written API, geared towards web developers
	\end{itemize}

