\subsection{Mobile technologies}

\subsubsection{Mobile platform}

\paragraph{Android}

	Android is an operating system developed for a variety of mobile devices.
	It is based on the linux kernel, and provides a developer-friendly
	framework for app development (subset of Java, Dalvik virtual machine).
	Most of it's code is released under the Apache Licence, a free software
	licence.

	Android was listed as the best-selling smartphone platform worldwide in Q4
	2010 by Canalys. \cite{wiki:android}

	\subparagraph{Pros}
		\begin{itemize}
			\item Can be developed using any major operating system (linux, windows, ...)
			\item Android devices can be emulated (AVD). This allows for testing on
			different screen sizes, etc.
			\item Developer-friendly
			\item Open source
		\end{itemize}

	\subparagraph{Cons}
		\begin{itemize}
			\item Apps developed for Android will not run on iOS or Windows phones
		\end{itemize}

\paragraph{iOS}

	iOS is Apple's mobile operating system, originally developed for the
	iPhone. Apple does not license iOS for third-party hardware.

	\subparagraph{Pros}
	\begin{itemize}
		\item Uniform screen size over most iPhone devices
		\item Design guidelines
	\end{itemize}

	\subparagraph{Cons}
		\begin{itemize}
			\item Closed and proprietary
			\item Apple hardware and software is required to develop iOS apps
			\item Requires a yearly subscription to distribute apps developed
			for iOS
			\item Apple maintains the right to remotely disable or delete apps
			at will
			\item GPL and other free software licenses can conflict with
			Apple's terms
			\item Problematic programming language, Objective-C (lacking garbage collector)
			\item Design guidelines can create artificial limitations
			\item Apps developed for iOS will only run on Apple products
		\end{itemize}

\subsubsection{Cross-compiling frameworks}
\paragraph{\bf{Phonegap}}
PhoneGap is an HTML5 app platform that allows you to author native applications with web technologies and get access to APIs and app stores. PhoneGap leverages web technologies such as HTML and JavaScript. PhoneGap is the only app platform available today that can publish to 6 platforms.
\cite{phonegap:about}
	\begin{itemize}
		\item Applications can be developed for Apple’s iOS, Google’s Android,
		Microsoft’s Windows Mobile, Nokia’s Symbian OS, RIM’s BlackBerry and Bada.
		\item Enables developers to take advantage of JavaScript, HTML5 and CSS3,
		which they might have already been familiar with.
		\item Access native features such as compass, camera, network, media,
		notifications, sound, vibrate and storage etc.
    \item Can use existing CSS and Javascript libraries directly in your code
    \item Seems like a native application when in reality it's an offline web application
    \item Provides a build tool for automatically building binary application packages for six different platforms
    \item Provided under the (new) BSD license or alternativly the MIT licence, the framework is entierly Open Source and free for Open Source projects.
    \item Provides a well-written API, geared towards web developers
	\end{itemize}

\paragraph{Corona}
Corona enables to develop graphically rich multimedia applications based on the programming
language Lua\cite{corona:about}. Corona supports both the iOS and Android platforms. Corona focuses on
applications with a lot of graphical animations. Corona is neither open source nor free, and also currently a developer
has to pay \$99 yearly to maintain membership and be able to build applications for App Store.

\paragraph{Titanium}
Titanium is a cross-platform mobile applications development framework that uses web
development technologies to develop applications for the Android and iOS mobile platforms.
Titanium is an open source framework that uses JavaScript and JSON as application language;
it can also use Python, Ruby and PHP scripts\cite{titanium:about}

\subsubsection{Native}
	Bulding applications natively is the best option if we look at each device
	separately. This ensures that your application always works, and you can
	utilize all the tools provided to you in the native API. With native
	applications you can also build applications more in conform with design
	guidelines for your platform, with native elements. This increases
	usability.

	\cite{phonegap:about}

	\begin{itemize}
		\item Must develop for each individual platform. This means different
		languages and APIs for every platform.
		\item Can utilize design guidelines for individual platforms, making
		apps similar to the user, increasing usability.
		\item Can access more functions in the API, creating cutting edge
		applications with the newest APIs
		\item Native applications runs faster and better on the phone
	\end{itemize}

