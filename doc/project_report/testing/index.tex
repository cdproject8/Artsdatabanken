\section{Testing}
\label{sec:testresult}

One of the key tasks that have been planned in this project was testing. In
section \ref{sec:testing}, a detailed purpose, plan and focus of our testing
strategy has been outlined. Having finished the prototype, it is very important
that each requirements were tested to make sure that no functionality
misbehaves or fails to perform as expected. Each of the nine requirements were
tested and the test results are documented in this section. This section
somewhat describes what has happened with the pilot prototype, a number of
modification have been incorporated as the testing went on to correct defects.
Several tests have been conducted on each functionality and code section during
the sprints, but this section is the final test result documentation that
declares each requirement as either working or not working.

The result of usability testing is also included as part of this section. The
usability test section tries to show how our prototype fared in the user space
of actual Artsdatabanken users. It gives a preview of how the application has
been received, criticized and modification prompts that arise from users.

\subsection{Functionality test results}

\input{testing/run1.tex}
\newpage
\subsection*{Test 1 (run 2)}

	\begin{table}[htb]
		\centering
		\begin{tabular}{|p{3.5cm}|p{7.0cm}|} \hline
			\textbf{Requirements} & F1 and F10 \\ \hline
			\textbf{Version} & 1.0 \\ \hline
			\textbf{Date} & 2011-11-08 \\ \hline
			\textbf{Tested by} & Muhsin Günaydin \\ \hline
			\textbf{Test environment} & HTC Desire running Android version 2.3.7 with kernel 2.6.37.6 \\ \hline
			\textbf{Pre-conditions} & Clean install of app on mobile device, no observations stored \\ \hline
			\textbf{Post-conditions} & A new observation has been saved and the user is directed back to the main menu \\ \hline
			\textbf{Result} & PASS \\ \hline
			\textbf{Comments} & Could only choose GPS coordinates in step 3, close locations missing \\ \hline
		\end{tabular}
		\caption{Summary of test 1 (run 2)}
	\end{table}

	\begin{table}[htb]
		\centering
		\begin{tabular}{|p{5.0cm}|p{5.0cm}|p{1cm}|}
			\hline \textbf{Action} & \textbf{Expected outcome} & \textbf{Result} \\ \hline
			1. Tap the new observation button & Menu for selecting species group appears & PASS \\ \hline
			2. Select species type "Fugl" (bird) & Menu for bird observations appears & PASS \\ \hline
			3. Selects location from list of close locations, or selects GPS location & Location is set & PASS \\ \hline
			4. Start writing "grågås" in the "Art" (species) field, ensure that
			auto-complete give useful suggestions, choose "grågås" from list of
			suggestions. Write 2 in the "Antall" (count) field & Auto-complete
			suggests bird names, 2 species of type "grågås" is added to observation
			& PASS \\ \hline 
			5. Tap "Lagre". & Observatin is saved & PASS \\ \hline
		\end{tabular}
		\caption{Execution of test 1 (run 2)}
	\end{table}

\newpage
\subsection*{Test 2 (run 2)}

	\begin{table}[htb]
		\centering
		\begin{tabular}{|p{3.5cm}|p{7.0cm}|} \hline
			\textbf{Requirement} & F2 \\ \hline
			\textbf{Version} & 1.0 \\ \hline
			\textbf{Date} & 2011-11-08 \\ \hline
			\textbf{Tested by} & Muhsin Günaydin \\ \hline
			\textbf{Test environment} & HTC Desire running Android version 2.3.7 with kernel 2.6.37.6 \\ \hline
			\textbf{Pre-conditions} & Test 1 completed in same test environment, app is still in observation view \\ \hline
			\textbf{Post-conditions} & Additional information about an observation has been saved \\ \hline
			\textbf{Result} & PASS \\ \hline
		\end{tabular}
		\caption{Summary of test 2 (run 2)}
	\end{table}

	\begin{table}[htb]
		\centering

		\begin{tabular}{|p{5.0cm}|p{5.0cm}|p{1cm}|}
			\hline \textbf{Action} & \textbf{Expected outcome} & \textbf{Result} \\ \hline

			1. Tap "Detaljer" (details) in the row containing "grågås" &
			Detailed view for the "grågås" observation is displayed & 
			PASS \\ \hline

			2. Select "Rugende" in the "Aktivitet" (activity) field &
			"Aktivitet" field populated with "Rugende" &
			PASS \\ \hline

			3. Tap back button & 
			Main observation view is displayed & 
			PASS \\ \hline

			4. Tap save and go to main screen. Go into the observatin using
			"Lagrede Observasjoner" (stored observations) and verify that "grågås"
			still has "Aktivitet" set to "Rugende" in the details view &
			Grågås has "Aktivitet" set to "Rugende" &
			PASS \\ \hline
		\end{tabular}
		\caption{Execution of test 2 (run 2)}
	\end{table}

\newpage
\subsection*{Test 3 (run 2)}

	\begin{table}[htb]
		\centering
		\begin{tabular}{|p{3.5cm}|p{7.0cm}|} \hline
			\textbf{Requirement} & F3 \\ \hline
			\textbf{Version} & 1.0 \\ \hline
			\textbf{Date} & 2011-11-08 \\ \hline
			\textbf{Tested by} & Muhsin Günaydin \\ \hline
			\textbf{Test environment} & HTC Desire running Android version 2.3.7 with kernel 2.6.37.6 \\ \hline
			\textbf{Pre-conditions} & Test 1 completed in same test environment, app is still in observation view \\ \hline
			\textbf{Post-conditions} & Observation is stored with two entries, "grågås" ("Antall" of 2) and "blåmeis" ("Antall" of 1) \\ \hline
			\textbf{Result} & PASS \\ \hline
			\textbf{Comments} & Step two not included  \\ \hline
		\end{tabular}
		\caption{Summary of test 3 (run 2)}
	\end{table}

	\begin{table}[htb]
		\centering
		\begin{tabular}{|p{5.0cm}|p{5.0cm}|p{1cm}|}
			\hline \textbf{Action} & \textbf{Expected outcome} & \textbf{Result} \\ \hline

			1. Tap "Legg til ny art" &
			A empty species row is appended &
			PASS \\ \hline
			
			2. Optionally selects another location, otherwise the same one is
			selected. & 
			New location chosen for current row &
			- \\ \hline

			3. Select "blåmeis" with the count ("Antall") of 1 in the same matter 
			as in test 1. &
			Row 2 is filled inn with ("Artsnavn" = "blåmes", "Antall" = 1) &
			PASS \\ \hline

			4. Tap "Lagre" &
			Observation is saved with two species observations ("grågås" and "blåmeis") &
			PASS \\ \hline
		\end{tabular}
		\caption{Execution of test 3 (run 2)}
	\end{table}

\newpage
\subsection*{Test 4 (run 2)}

	\begin{table}[htb]
		\centering
		\begin{tabular}{|p{3.5cm}|p{7.0cm}|} \hline
			\textbf{Requirement} & F4 \\ \hline
			\textbf{Version} & 1.0 \\ \hline
			\textbf{Date} & 2011-11-08 \\ \hline
			\textbf{Tested by} & Muhsin Günaydin \\ \hline
			\textbf{Test environment} & HTC Desire running Android version 2.3.7 with kernel 2.6.37.6 \\ \hline
			\textbf{Pre-conditions} & Test 2 and 3 completed in same test environment, app is still in observation view \\ \hline
			\textbf{Post-conditions} & All data from current observation is submitted to the native mail client \\ \hline
			\textbf{Result} & PASS \\ \hline
			\textbf{Comments} & The exported text is not detailed  \\ \hline
		\end{tabular}
		\caption{Summary of test 4 (run 2)}
	\end{table}

	\begin{table}[htb]
		\centering
		\begin{tabular}{|p{5.0cm}|p{5.0cm}|p{1cm}|}
			\hline \textbf{Action} & \textbf{Expected outcome} & \textbf{Result} \\ \hline
			1. Taps 'Eksporter' (export) in the observation view &
			Native email client is launched in "new email"-mode. Date from
			observation is placed in the message field. &
			PASS \\ \hline
		\end{tabular}
		\caption{Execution of test 4 (run 2)}
	\end{table}

\newpage
\subsection*{Test 5 (run 2)}

	\begin{table}[htb]
		\centering
		\begin{tabular}{|p{3.5cm}|p{7.0cm}|} \hline
			\textbf{Requirement} & F5 \\ \hline
			\textbf{Version} & 1.0 \\ \hline
			\textbf{Date} & 2011-11-08 \\ \hline
			\textbf{Tested by} & Muhsin Günaydin \\ \hline
			\textbf{Test environment} & HTC Desire running Android version 2.3.7 with kernel 2.6.37.6 \\ \hline
			\textbf{Pre-conditions} & App installed on mobile device \\ \hline
			\textbf{Post-conditions} & Picture is stored on the phone with an easily recognizable filename \\ \hline
			\textbf{Result} & PASS \\ \hline
			\textbf{Comments} & Can choose from earlier taken/saved photos, cant take a new picture. It wont be usefull to take a picture which is not assosiated with an observation, the take picture button in the main window should be removed .    \\ \hline
		\end{tabular}
		\caption{Summary of test 5 (run 2)}
	\end{table}

	\begin{table}[htb]
		\centering
		\begin{tabular}{|p{5.0cm}|p{5.0cm}|p{1cm}|}
			\hline \textbf{Action} & \textbf{Expected outcome} & \textbf{Result} \\ \hline
			1. Tap "Ta Bilde" (capture image) in the main view &
			Native image capturing software is started & 
			- \\ \hline

			2. Take picture using native software &
			Image is stored and success message is displayed in app &
			-\\ \hline
		\end{tabular}
		\caption{Execution of test 5 (run 2)}
	\end{table}

\newpage
\subsection*{Test 6 (run 2)}

	\begin{table}[htb]
		\centering
		\begin{tabular}{|p{3.5cm}|p{7.0cm}|} \hline
			\textbf{Requirement} & F6 \\ \hline
			\textbf{Version} & 0.9 \\ \hline
			\textbf{Date} & 2011-11-08 \\ \hline
			\textbf{Tested by} & Muhsin Günaydin \\ \hline
			\textbf{Test environment} & HTC Desire running Android version 2.3.7 with kernel 2.6.37.6 \\ \hline
			\textbf{Pre-conditions} & Test 2 completed \\ \hline
			\textbf{Post-conditions} & App is in same state as before test started \\ \hline
			\textbf{Result} & PASS \\ \hline
			\textbf{Comments} & \\ \hline
		\end{tabular}
		\caption{Summary of test 6 (run 2)}
	\end{table}

	\begin{table}[htb]
		\centering
		\begin{tabular}{|p{5.0cm}|p{5.0cm}|p{1cm}|}
			\hline \textbf{Action} & \textbf{Expected outcome} & \textbf{Result} \\ \hline
				1. Tap "Lagrede Observasjoner" from the main view &
				A list containing one observatin (from test 2) is displayed &
				PASS \\ \hline

				2. Select observation 1 from the list &
				The observation from test 2 is displayed in the same state as
				earlier (two species) &
				PASS \\ \hline

				3. Tap "Detailer" in the row containing "grågås" &
				Details view for "grågås" observation is displayed, the field "Aktivitet"
				is filled in with "Rugende" &
				PASS \\ \hline
		\end{tabular}
		\caption{Execution of test 6 (run 2)}
	\end{table}

\newpage
\subsection*{Test 7 (run 2)}

	\begin{table}[htb]
		\centering
		\begin{tabular}{|p{3.5cm}|p{7.0cm}|} \hline
			\textbf{Requirement} & F7 \\ \hline
			\textbf{Version} & 0.9 \\ \hline
			\textbf{Date} & 2011-11-08 \\ \hline
			\textbf{Tested by} & Muhsin Günaydin \\ \hline
			\textbf{Test environment} & HTC Desire running Android version 2.3.7 with kernel 2.6.37.6 \\ \hline
			\textbf{Pre-conditions} & Test 6 completed, still viewing stored observation \\ \hline
			\textbf{Post-conditions} & Observation is stored on device with an additional row containing ("Art" = "grønnfink", "Antall" = 9) \\ \hline
			\textbf{Result} & PASS \\ \hline
			\textbf{Comments} & \\ \hline
		\end{tabular}
		\caption{Summary of test 7 (run 2)}
	\end{table}

	\begin{table}[htb]
		\centering
		\begin{tabular}{|p{5.0cm}|p{5.0cm}|p{1cm}|}
			\hline \textbf{Action} & \textbf{Expected outcome} & \textbf{Result} \\ \hline

			1. Add a new row with the same procedure as in test 3 &
			A empty row is appended to the observation &
			PASS \\ \hline

			2. Fill in ("Art" = "grønnfink" and "Antall" = 9) in the new row
			using the same procedure as in test 2. &
			New row is populated with ("grønnfink", 9) &
			PASS \\ \hline

			3. Tap "Lagre" &
			Observation stored with an additional row ("grønnfink", 9) &
			PASS \\ \hline

		\end{tabular}
		\caption{Execution of test 7 (run 2)}
	\end{table}

\newpage
\subsection*{Test 8 (run 2)}

	\begin{table}[htb]
		\centering
		\begin{tabular}{|p{3.5cm}|p{7.0cm}|} \hline
			\textbf{Requirement} & F9 \\ \hline
			\textbf{Version} & 0.9 \\ \hline
			\textbf{Date} & 2011-11-08 \\ \hline
			\textbf{Tested by} & Muhsin Günaydin \\ \hline
			\textbf{Test environment} & HTC Desire running Android version 2.3.7 with kernel 2.6.37.6 \\ \hline
			\textbf{Pre-conditions} & Test 1 completed, still viewing stored observation \\ \hline
			\textbf{Post-conditions} App is in same state as before test execution &  \\ \hline
			\textbf{Result} & PASS \\ \hline
			\textbf{Comments} & Got the fields updated although the GPS was off. Maybe the location was saved using WLAN/3G..  \\
			\hline
		\end{tabular}
		\caption{Summary of test 8 (run 2)}
	\end{table}

	\begin{table}[htb]
		\centering
		\begin{tabular}{|p{5.0cm}|p{5.0cm}|p{1cm}|}
			\hline \textbf{Action} & \textbf{Expected outcome} & \textbf{Result} \\ \hline
			
			1. Tap "GPS" &
			Longitude and latitude contains coordinates near the current location &
			PASS \\ \hline

		\end{tabular}
		\caption{Execution of test 8 (run 2)}
	\end{table}

\newpage

\newpage
\subsection{Test 1 (run 3)}

	\begin{figure}[htb]
		\centering
		\begin{tabular}{|p{3.5cm}|p{7.0cm}|} \hline
			\textbf{Requirements} & F1 and F10 \\ \hline
			\textbf{Version} & 1.0 \\ \hline
			\textbf{Date} & 2011-11-08 \\ \hline
			\textbf{Tested by} & Stian Liknes \\ \hline
			\textbf{Test environment} & Sony Ericsson Xperia X10 running Android 2.1.1.A.0.6 with kernel: 2.6.29 \\ \hline
			\textbf{Pre-conditions} & Clean install of app on mobile device, no observations stored \\ \hline
			\textbf{Post-conditions} & A new observation has been saved and the user is directed back to the main menu \\ \hline
			\textbf{Result} & PASS \\ \hline
		\end{tabular}
		\caption{Summary of test 1}
	\end{figure}

	\begin{figure}[htb]
		\centering
		\begin{tabular}{|p{5.0cm}|p{5.0cm}|p{1cm}|}
			\hline \textbf{Action} & \textbf{Expected outcome} & \textbf{Result} \\ \hline
			1. Tap the new observation button & Menu for selecting species group appears & PASS \\ \hline
			2. Select species type "Fugl" (bird) & Menu for bird observations appears & PASS \\ \hline
			3. Selects location from list of close locations, or selects GPS location & Location is set & PASS \\ \hline
			4. Start writing "grågås" in the "Art" (species) field, ensure that
			auto-complete give useful suggestions, choose "grågås" from list of
			suggestions. Write 2 in the "Antall" (count) field & Auto-complete
			suggests bird names, 2 species of type "grågås" is added to observation
			& PASS \\ \hline 
			5. Tap "Lagre". & Observatin is saved & PASS \\ \hline
		\end{tabular}
		\caption{Execution of test 1}
	\end{figure}

\newpage
\subsection{Test 2 (run 3)}

	\begin{figure}[htb]
		\centering
		\begin{tabular}{|p{3.5cm}|p{7.0cm}|} \hline
			\textbf{Requirement} & F2 \\ \hline
			\textbf{Version} & 1.0 \\ \hline
			\textbf{Date} & 2011-11-08 \\ \hline
			\textbf{Tested by} & Stian Liknes \\ \hline
			\textbf{Test environment} & Sony Ericsson Xperia X10 running Android 2.1.1.A.0.6 with kernel: 2.6.29 \\ \hline
			\textbf{Pre-conditions} & Test 1 completed in same test environment, app is still in observation view \\ \hline
			\textbf{Post-conditions} & Additional information about an observation has been saved \\ \hline
			\textbf{Result} & PASS \\ \hline
		\end{tabular}
		\caption{Summary of test 2}
	\end{figure}

	\begin{figure}[htb]
		\centering

		\begin{tabular}{|p{5.0cm}|p{5.0cm}|p{1cm}|}
			\hline \textbf{Action} & \textbf{Expected outcome} & \textbf{Result} \\ \hline

			1. Tap "Detaljer" (details) in the row containing "grågås" &
			Detailed view for the "grågås" observation is displayed & 
			PASS \\ \hline

			2. Select "Rugende" in the "Aktivitet" (activity) field &
			"Aktivitet" field populated with "Rugende" &
			PASS \\ \hline

			3. Tap back button & 
			Main observation view is displayed & 
			PASS \\ \hline

			4. Tap save and go to main screen. Go into the observatin using
			"Lagrede Observasjoner" (stored observations) and verify that "grågås"
			still has "Aktivitet" set to "Rugende" in the details view &
			Grågås has "Aktivitet" set to "Rugende" &
			PASS \\ \hline
		\end{tabular}
		\caption{Execution of test 2}
	\end{figure}

\newpage
\subsection{Test 3 (run 3)}

	\begin{figure}[htb]
		\centering
		\begin{tabular}{|p{3.5cm}|p{7.0cm}|} \hline
			\textbf{Requirement} & F3 \\ \hline
			\textbf{Version} & 1.0 \\ \hline
			\textbf{Date} & 2011-11-08 \\ \hline
			\textbf{Tested by} & Stian Liknes \\ \hline
			\textbf{Test environment} & Sony Ericsson Xperia X10 running Android 2.1.1.A.0.6 with kernel: 2.6.29 \\ \hline
			\textbf{Pre-conditions} & Test 1 completed in same test environment, app is still in observation view \\ \hline
			\textbf{Post-conditions} & Observation is stored with two entries, "grågås" ("Antall" of 2) and "blåmeis" ("Antall" of 1) \\ \hline
			\textbf{Result} & PASS \\ \hline
			\textbf{Comments} & We decided to remove step 2 from the use case \\ \hline
		\end{tabular}
		\caption{Summary of test 3}
	\end{figure}

	\begin{figure}[htb]
		\centering
		\begin{tabular}{|p{5.0cm}|p{5.0cm}|p{1cm}|}
			\hline \textbf{Action} & \textbf{Expected outcome} & \textbf{Result} \\ \hline

			1. Tap "Legg til ny art" &
			A empty species row is appended &
			PASS \\ \hline
			
			2. Optionally selects another location, otherwise the same one is
			selected. & 
			New location chosen for current row &
			- \\ \hline

			3. Select "blåmeis" with the count ("Antall") of 1 in the same matter 
			as in test 1. &
			Row 2 is filled inn with ("Artsnavn" = "blåmes", "Antall" = 1) &
			PASS \\ \hline

			4. Tap "Lagre" &
			Observation is saved with two species observations ("grågås" and "blåmeis") &
			PASS \\ \hline
		\end{tabular}
		\caption{Execution of test 3}
	\end{figure}

\newpage
\subsection{Test 4 (run 3)}

	\begin{figure}[htb]
		\centering
		\begin{tabular}{|p{3.5cm}|p{7.0cm}|} \hline
			\textbf{Requirement} & F4 \\ \hline
			\textbf{Version} & 1.0 \\ \hline
			\textbf{Date} & 2011-11-08 \\ \hline
			\textbf{Tested by} & Stian Liknes \\ \hline
			\textbf{Test environment} & Sony Ericsson Xperia X10 running Android 2.1.1.A.0.6 with kernel: 2.6.29 \\ \hline
			\textbf{Pre-conditions} & Test 2 and 3 completed in same test environment, app is still in observation view \\ \hline
			\textbf{Post-conditions} & All data from current observation is submitted to the native mail client \\ \hline
			\textbf{Result} & PASS \\ \hline
		\end{tabular}
		\caption{Summary of test 4}
	\end{figure}

	\begin{figure}[htb]
		\centering
		\begin{tabular}{|p{5.0cm}|p{5.0cm}|p{1cm}|}
			\hline \textbf{Action} & \textbf{Expected outcome} & \textbf{Result} \\ \hline
			1. Taps 'Eksporter' (export) in the observation view &
			Native email client is launched in "new email"-mode. Date from
			observation is placed in the message field. &
			PASS \\ \hline
		\end{tabular}
		\caption{Execution of test 4}
	\end{figure}

\newpage
\subsection{Test 5 (run 3)}

	\begin{figure}[htb]
		\centering
		\begin{tabular}{|p{3.5cm}|p{7.0cm}|} \hline
			\textbf{Requirement} & F5 \\ \hline
			\textbf{Version} & 1.0 \\ \hline
			\textbf{Date} & 2011-11-08 \\ \hline
			\textbf{Tested by} & Stian Liknes \\ \hline
			\textbf{Test environment} & Sony Ericsson Xperia X10 running Android 2.1.1.A.0.6 with kernel: 2.6.29 \\ \hline
			\textbf{Pre-conditions} & App installed on mobile device \\ \hline
			\textbf{Post-conditions} & Picture is stored on the phone with an easily recognizable filename \\ \hline
			\textbf{Result} & FAIL \\ \hline
			\textbf{Comments} & Could only take picture when no other pictures existed on device. Othewise you can select existing images.\\ \hline
		\end{tabular}
		\caption{Summary of test 5}
	\end{figure}

	\begin{figure}[htb]
		\centering
		\begin{tabular}{|p{5.0cm}|p{5.0cm}|p{1cm}|}
			\hline \textbf{Action} & \textbf{Expected outcome} & \textbf{Result} \\ \hline
			1. Tap "Ta Bilde" (capture image) in the main view &
			Native image capturing software is started & 
			PASS \\ \hline

			2. Take picture using native software &
			Image is stored and success message is displayed in app &
			FAIL \\ \hline
		\end{tabular}
		\caption{Execution of test 5}
	\end{figure}

\newpage
\subsection{Test 6 (run 3)}

	\begin{figure}[htb]
		\centering
		\begin{tabular}{|p{3.5cm}|p{7.0cm}|} \hline
			\textbf{Requirement} & F6 \\ \hline
			\textbf{Version} & 1.0 \\ \hline
			\textbf{Date} & 2011-11-08 \\ \hline
			\textbf{Tested by} & Stian Liknes \\ \hline
			\textbf{Test environment} & Sony Ericsson Xperia X10 running Android 2.1.1.A.0.6 with kernel: 2.6.29 \\ \hline
			\textbf{Pre-conditions} & Test 2 completed \\ \hline
			\textbf{Post-conditions} & App is in same state as before test started \\ \hline
			\textbf{Result} & PASS \\ \hline
		\end{tabular}
		\caption{Summary of test 6}
	\end{figure}

	\begin{figure}[htb]
		\centering
		\begin{tabular}{|p{5.0cm}|p{5.0cm}|p{1cm}|}
			\hline \textbf{Action} & \textbf{Expected outcome} & \textbf{Result} \\ \hline
				1. Tap "Lagrede Observasjoner" from the main view &
				A list containing one observatin (from test 2) is displayed &
				PASS \\ \hline

				2. Select observation 1 from the list &
				The observation from test 2 is displayed in the same state as
				earlier (two species) &
				PASS \\ \hline

				3. Tap "Detailer" in the row containing "grågås" &
				Details view for "grågås" observation is displayed, the field "Aktivitet"
				is filled in with "Rugende" &
				PASS \\ \hline
		\end{tabular}
		\caption{Execution of test 6}
	\end{figure}

\newpage
\subsection{Test 7 (run 3)}

	\begin{figure}[htb]
		\centering
		\begin{tabular}{|p{3.5cm}|p{7.0cm}|} \hline
			\textbf{Requirement} & F7 \\ \hline
			\textbf{Version} & 1.0 \\ \hline
			\textbf{Date} & 2011-11-08 \\ \hline
			\textbf{Tested by} & Stian Liknes \\ \hline
			\textbf{Test environment} & Sony Ericsson Xperia X10 running Android 2.1.1.A.0.6 with kernel: 2.6.29 \\ \hline
			\textbf{Pre-conditions} & Test 6 completed, still viewing stored observation \\ \hline
			\textbf{Post-conditions} & Observation is stored on device with an additional row containing ("Art" = "grønnfink", "Antall" = 9) \\ \hline
			\textbf{Result} & PASS \\ \hline
		\end{tabular}
		\caption{Summary of test 7}
	\end{figure}

	\begin{figure}[htb]
		\centering
		\begin{tabular}{|p{5.0cm}|p{5.0cm}|p{1cm}|}
			\hline \textbf{Action} & \textbf{Expected outcome} & \textbf{Result} \\ \hline

			1. Add a new row with the same procedure as in test 2 &
			A empty row is appended to the observation &
			PASS \\ \hline

			2. Fill in ("Art" = "grønnfink" and "Antall" = 9) in the new row
			using the same procedure as in test 2. &
			New row is populated with ("grønnfink", 9) &
			PASS \\ \hline

			3. Tap "Lagre" &
			Observation stored with an additional row ("grønnfink", 9) &
			PASS \\ \hline

		\end{tabular}
		\caption{Execution of test 7}
	\end{figure}

\newpage
\subsection{Test 8 (run 3)}

	\begin{figure}[htb]
		\centering
		\begin{tabular}{|p{3.5cm}|p{7.0cm}|} \hline
			\textbf{Requirement} & F9 \\ \hline
			\textbf{Version} & 1.0 \\ \hline
			\textbf{Date} & 2011-11-08 \\ \hline
			\textbf{Tested by} & Stian Liknes \\ \hline
			\textbf{Test environment} & Sony Ericsson Xperia X10 running Android 2.1.1.A.0.6 with kernel: 2.6.29 \\ \hline
			\textbf{Pre-conditions} & Test 1 completed, still viewing stored observation \\ \hline
			\textbf{Post-conditions} & App is in same state as before test execution \\ \hline
			\textbf{Result} & PASS \\ \hline
		\end{tabular}
		\caption{Summary of test 8}
	\end{figure}

	\begin{figure}[htb]
		\centering
		\begin{tabular}{|p{5.0cm}|p{5.0cm}|p{1cm}|}
			\hline \textbf{Action} & \textbf{Expected outcome} & \textbf{Result} \\ \hline
			
			1. Tap "GPS" &
			Longitude and latitude contains coordinates near the current location &
			PASS \\ \hline

		\end{tabular}
		\caption{Execution of test 8}
	\end{figure}

\newpage


\subsection*{Test 5 Fixed (run 1)}

	\begin{table}[htb]
		\centering
		\begin{tabular}{|p{3.5cm}|p{7.0cm}|} \hline
			\textbf{Requirement} & F5 \\ \hline
			\textbf{Version} & 1.0 \\ \hline
			\textbf{Date} & 2011-11-08 \\ \hline
			\textbf{Tested by} & Muhsin Günaydin \\ \hline
			\textbf{Test environment} & HTC Desire running Android version 2.3.7 with kernel 2.6.37.6 \\ \hline
			\textbf{Pre-conditions} & App installed on mobile device \\ \hline
			\textbf{Post-conditions} & Picture is stored on the phone with an easily recognizable filename \\ \hline
			\textbf{Result} & PASS \\ \hline
			\textbf{Comments} & \\ \hline
		\end{tabular}
		\caption{Summary of test 5 (run 1)}
	\end{table}

	\begin{table}[htb]
		\centering
		\begin{tabular}{|p{5.0cm}|p{5.0cm}|p{1cm}|}
			\hline \textbf{Action} & \textbf{Expected outcome} & \textbf{Result} \\ \hline
			1. Tap "Detaljer" (details) in the row containing "grågås"  &
			Detailed view for the "grågås" observation is displayed & 
			PASS \\ \hline

			2. Tap "Ta bilde" (take picture) at the bottom &
			Native image capturing software is started &
			PASS\\ \hline

			3. Take picture using native software &
			Image is captured and displayed in details for the Art &
			PASS\\ \hline
			
			4. Tap "Hent Bilde" (get picture) &
			Native file explorer software is started &
			PASS\\ \hline

			5. Select picture from the native software &
			The selected picture is displayed under the captured pictured &
			PASS\\ \hline
		
		\end{tabular}
		\caption{Execution of test 5 (run 1)}
	\end{table}


\newpage


\subsection{Usability test results}

Artsdatabanken contracted a company to carry out a professional usability
testing for the prototype we shipped. Artsdatabanken preferred to conduct the
usability test by itself and report the results back to us. However, our
customer managed to sign up only 3 participants for the usability test, a participant number
which is significantly lower than our minimum specification of 20 participants.
The plan was to analyse the results using standard statistical result analysis
methods---showing the preference, skill and acceptance of the application which
would have been reflected by a reasonable sample size. When, in statistics,
bigger sample size gives better results\cite{statistics}, deriving any
conclusions from a sample size of three will be unrealistic and the conclusion we draw, if
any, will be difficult to rely on. Below are some of the result summarizations we managed to collect from three of the participants. You can find the usability form on http://bit.ly/rHDBop

Our usability test was conducted using some sections to test the background of participants and their overall experience of the application.
Refer to the following color legend keys to interpret and understand the charts coming below.

\begin{figure}[htb]
\centering
 \begin{center}$
 \begin{array}{cc}
 \includegraphics[scale=0.94]{ut_pic/legenda.png} &
 \includegraphics[scale=0.9]{ut_pic/legend3.png}
 \includegraphics[scale=0.8]{ut_pic/legend2.png}
 \end{array}$
 \end{center}
 \caption{Legend colour schemes. For general opinion and agreement (Left), Functionality like and dislike (center), Mobile Application experience}
 \end{figure}


In the first section, we tried to understand who the participants were so we can divide and analyse them by their differences age and degree of Mobile App and Smartphone experiences. We found out that all of the three participants are in the age group of 33-44, all of them are also experts in general web application and owned an Android based smart-phones but a little variation on Mobile App experience.

\begin{figure}[htb]
\centering
 \begin{center}$
 \begin{array}{cc}
 \includegraphics[scale=0.7]{ut_pic/mobileapp1.jpg}
 
 \label{Mobile Applicatoin Experience}
  \end{array}$
 \end{center}
 \caption{Mobile application Experience.}
 \end{figure}



The figure shows that Mobile App experience of the participants seem to be split equally in Moderate, Good and Expert level of experience. The fact that all of them are in the same age group makes it harder to forecast how the application is going to fare among the general real user populations. The usability test was conducted by participants who owned or had only used Android based devices and nothing can be said about how the application would have behaved or received by users from the Apple's iOS platform.

The second section of our usability test was to find out how the prototype has productivity, performance, intention to use, learnability, user friendliness, clarity and popularity of functionalities.

All of the participants found the application easy to learn, easy to use and that they would recommend it to any new user. Using questionnaire items, we seem to notice that there is some contradiction between the application being easy to set but 2 out of 3 participants found it troublesome to install. We need a bigger sample to get a clarify this little contradiction that will actually help to improve the application.

\begin{figure}[htb]
    \centering
    \includegraphics[scale=0.7]{ut_pic/difficutltoinstall1.jpg}
    \label{Application installation}
    \caption{I found the App difficult to install}
\end{figure}

In general, the results seem to show that the application has fared positively on average. Although it is very difficult to draw any conclusion with certainty figures, the next figures show the different opinions of the usability test participants.

\begin{figure}[htb]
\centering
 \begin{center}$
 \begin{array}{cc}
 \includegraphics[scale=0.5]{ut_pic/easytosetup1.jpg} &
 \includegraphics[scale=0.5]{ut_pic/hasmetmyexpectation1.jpg}
 \end{array}$
 \end{center}
 \caption{The App is easy to set up (left), Majority neutral for the App has met my expectation (right)}
 \end{figure}
 
 \begin{figure}[htb]
     \centering
     \includegraphics[scale=0.7]{ut_pic/userfriendlybuttons1.jpg}
     \label{fig:User friendly buttons}
     \caption{The App has user-friendly interface}
 \end{figure}
 
 In terms of productivity and intention to use, the majority seem to think that the application will make them productive and they also intend to use it in the future.
 
\begin{figure}[htb]
    \centering
    \includegraphics[scale=0.7]{ut_pic/productivityandintensiontouse1.jpg}
    \label{fig:Productivity and intention to use}
    \caption{Majority result showing positive result for productivity and intention to use}
\end{figure}

While all three of our participants liked the functionality New observation, each seem to have a least favourite functionality. Export, Take picture and Store observation have each one participant not favouring them.

\begin{figure}[htb]
    \centering
    \includegraphics[scale=0.7]{ut_pic/Functionalitydislike1.jpg}
    \label{fig:Functionality favourability}
    \caption{Least favourite functionality}
\end{figure}

For the questionnaire where the application was flawless and runs smoothly, two disagreed and one remained neutral.

\begin{figure}[htb]
    \centering
    \includegraphics[scale=0.6]{ut_pic/flawless1.jpg}
    \label{fig:App was flawless}
    \caption{Majority disagree on the application being flawless and running smoothly}
\end{figure}

This was the only question we put as an inverse of the rest of the questionnaire items to see if the application was found confusing to use. And disagreed on the application being confusing, which means the application was not confusing to use.

\begin{figure}[htb]
    \centering
    \includegraphics[scale=0.6]{ut_pic/confusing1.jpg}
    \label{fig:App was confusing to use}
    \caption{The application was found to be not confusing}
\end{figure}

\subsubsection{Summary}
With a better number of participants, we could split the participants based on age, experience and device to figure out how each group will behave using Friedman\cite{Friedman} and Wilcoxon\cite{Wilcoxon} Tests. User generation related concerns, device and platform specific problems, technology acceptance problems and other variety of problems will be easier to zone in on with an optimum sample size. The result from such analysis are much more reliable for suggesting improvements on future work for the prototype and it will also help us understand the users, and get insight as to how they perceived the application plus missing functionalities that could easily have been incorporated to make the application much more robust.
\newpage

