\newpage
\subsection*{Test 1 (run 3)}

	\begin{figure}[htb]
		\centering
		\begin{tabular}{|p{3.5cm}|p{7.0cm}|} \hline
			\textbf{Requirements} & F1 and F10 \\ \hline
			\textbf{Version} & 1.0 \\ \hline
			\textbf{Date} & 2011-11-08 \\ \hline
			\textbf{Tested by} & Stian Liknes \\ \hline
			\textbf{Test environment} & Sony Ericsson Xperia X10 running Android 2.1.1.A.0.6 with kernel: 2.6.29 \\ \hline
			\textbf{Pre-conditions} & Clean install of app on mobile device, no observations stored \\ \hline
			\textbf{Post-conditions} & A new observation has been saved and the user is directed back to the main menu \\ \hline
			\textbf{Result} & PASS \\ \hline
		\end{tabular}
		\caption{Summary of test 1}
	\end{figure}

	\begin{figure}[htb]
		\centering
		\begin{tabular}{|p{5.0cm}|p{5.0cm}|p{1cm}|}
			\hline \textbf{Action} & \textbf{Expected outcome} & \textbf{Result} \\ \hline
			1. Tap the new observation button & Menu for selecting species group appears & PASS \\ \hline
			2. Select species type "Fugl" (bird) & Menu for bird observations appears & PASS \\ \hline
			3. Selects location from list of close locations, or selects GPS location & Location is set & PASS \\ \hline
			4. Start writing "grågås" in the "Art" (species) field, ensure that
			auto-complete give useful suggestions, choose "grågås" from list of
			suggestions. Write 2 in the "Antall" (count) field & Auto-complete
			suggests bird names, 2 species of type "grågås" is added to observation
			& PASS \\ \hline 
			5. Tap "Lagre". & Observatin is saved & PASS \\ \hline
		\end{tabular}
		\caption{Execution of test 1}
	\end{figure}

\newpage
\subsection*{Test 2 (run 3)}

	\begin{figure}[htb]
		\centering
		\begin{tabular}{|p{3.5cm}|p{7.0cm}|} \hline
			\textbf{Requirement} & F2 \\ \hline
			\textbf{Version} & 1.0 \\ \hline
			\textbf{Date} & 2011-11-08 \\ \hline
			\textbf{Tested by} & Stian Liknes \\ \hline
			\textbf{Test environment} & Sony Ericsson Xperia X10 running Android 2.1.1.A.0.6 with kernel: 2.6.29 \\ \hline
			\textbf{Pre-conditions} & Test 1 completed in same test environment, app is still in observation view \\ \hline
			\textbf{Post-conditions} & Additional information about an observation has been saved \\ \hline
			\textbf{Result} & PASS \\ \hline
		\end{tabular}
		\caption{Summary of test 2}
	\end{figure}

	\begin{figure}[htb]
		\centering

		\begin{tabular}{|p{5.0cm}|p{5.0cm}|p{1cm}|}
			\hline \textbf{Action} & \textbf{Expected outcome} & \textbf{Result} \\ \hline

			1. Tap "Detaljer" (details) in the row containing "grågås" &
			Detailed view for the "grågås" observation is displayed & 
			PASS \\ \hline

			2. Select "Rugende" in the "Aktivitet" (activity) field &
			"Aktivitet" field populated with "Rugende" &
			PASS \\ \hline

			3. Tap back button & 
			Main observation view is displayed & 
			PASS \\ \hline

			4. Tap save and go to main screen. Go into the observatin using
			"Lagrede Observasjoner" (stored observations) and verify that "grågås"
			still has "Aktivitet" set to "Rugende" in the details view &
			Grågås has "Aktivitet" set to "Rugende" &
			PASS \\ \hline
		\end{tabular}
		\caption{Execution of test 2}
	\end{figure}

\newpage
\subsection*{Test 3 (run 3)}

	\begin{figure}[htb]
		\centering
		\begin{tabular}{|p{3.5cm}|p{7.0cm}|} \hline
			\textbf{Requirement} & F3 \\ \hline
			\textbf{Version} & 1.0 \\ \hline
			\textbf{Date} & 2011-11-08 \\ \hline
			\textbf{Tested by} & Stian Liknes \\ \hline
			\textbf{Test environment} & Sony Ericsson Xperia X10 running Android 2.1.1.A.0.6 with kernel: 2.6.29 \\ \hline
			\textbf{Pre-conditions} & Test 1 completed in same test environment, app is still in observation view \\ \hline
			\textbf{Post-conditions} & Observation is stored with two entries, "grågås" ("Antall" of 2) and "blåmeis" ("Antall" of 1) \\ \hline
			\textbf{Result} & PASS \\ \hline
			\textbf{Comments} & We decided to remove step 2 from the use case \\ \hline
		\end{tabular}
		\caption{Summary of test 3}
	\end{figure}

	\begin{figure}[htb]
		\centering
		\begin{tabular}{|p{5.0cm}|p{5.0cm}|p{1cm}|}
			\hline \textbf{Action} & \textbf{Expected outcome} & \textbf{Result} \\ \hline

			1. Tap "Legg til ny art" &
			A empty species row is appended &
			PASS \\ \hline
			
			2. Optionally selects another location, otherwise the same one is
			selected. & 
			New location chosen for current row &
			- \\ \hline

			3. Select "blåmeis" with the count ("Antall") of 1 in the same matter 
			as in test 1. &
			Row 2 is filled inn with ("Artsnavn" = "blåmes", "Antall" = 1) &
			PASS \\ \hline

			4. Tap "Lagre" &
			Observation is saved with two species observations ("grågås" and "blåmeis") &
			PASS \\ \hline
		\end{tabular}
		\caption{Execution of test 3}
	\end{figure}

\newpage
\subsection*{Test 4 (run 3)}

	\begin{figure}[htb]
		\centering
		\begin{tabular}{|p{3.5cm}|p{7.0cm}|} \hline
			\textbf{Requirement} & F4 \\ \hline
			\textbf{Version} & 1.0 \\ \hline
			\textbf{Date} & 2011-11-08 \\ \hline
			\textbf{Tested by} & Stian Liknes \\ \hline
			\textbf{Test environment} & Sony Ericsson Xperia X10 running Android 2.1.1.A.0.6 with kernel: 2.6.29 \\ \hline
			\textbf{Pre-conditions} & Test 2 and 3 completed in same test environment, app is still in observation view \\ \hline
			\textbf{Post-conditions} & All data from current observation is submitted to the native mail client \\ \hline
			\textbf{Result} & PASS \\ \hline
		\end{tabular}
		\caption{Summary of test 4}
	\end{figure}

	\begin{figure}[htb]
		\centering
		\begin{tabular}{|p{5.0cm}|p{5.0cm}|p{1cm}|}
			\hline \textbf{Action} & \textbf{Expected outcome} & \textbf{Result} \\ \hline
			1. Taps 'Eksporter' (export) in the observation view &
			Native email client is launched in "new email"-mode. Date from
			observation is placed in the message field. &
			PASS \\ \hline
		\end{tabular}
		\caption{Execution of test 4}
	\end{figure}

\newpage
\subsection*{Test 5 (run 3)}

	\begin{figure}[htb]
		\centering
		\begin{tabular}{|p{3.5cm}|p{7.0cm}|} \hline
			\textbf{Requirement} & F5 \\ \hline
			\textbf{Version} & 1.0 \\ \hline
			\textbf{Date} & 2011-11-08 \\ \hline
			\textbf{Tested by} & Stian Liknes \\ \hline
			\textbf{Test environment} & Sony Ericsson Xperia X10 running Android 2.1.1.A.0.6 with kernel: 2.6.29 \\ \hline
			\textbf{Pre-conditions} & App installed on mobile device \\ \hline
			\textbf{Post-conditions} & Picture is stored on the phone with an easily recognizable filename \\ \hline
			\textbf{Result} & PASS \\ \hline
			\textbf{Comments} & Uncovered a weakness in our user interface. The
			"Ta bilde" button is not used to capture images, we have an option
			for this under "Detaljer" for each observation. Could rename the
				"Ta bilde" button, or better yet, remove it.\\ \hline
		\end{tabular}
		\caption{Summary of test 5}
	\end{figure}

	\begin{figure}[htb]
		\centering
		\begin{tabular}{|p{5.0cm}|p{5.0cm}|p{1cm}|}
			\hline \textbf{Action} & \textbf{Expected outcome} & \textbf{Result} \\ \hline
			1. Tap "Ta Bilde" (capture image) in the main view &
			Native image capturing software is started & 
			PASS \\ \hline

			2. Take picture using native software &
			Image is stored and success message is displayed in app &
			- \\ \hline
		\end{tabular}
		\caption{Execution of test 5}
	\end{figure}

\newpage
\subsection*{Test 6 (run 3)}

	\begin{figure}[htb]
		\centering
		\begin{tabular}{|p{3.5cm}|p{7.0cm}|} \hline
			\textbf{Requirement} & F6 \\ \hline
			\textbf{Version} & 1.0 \\ \hline
			\textbf{Date} & 2011-11-08 \\ \hline
			\textbf{Tested by} & Stian Liknes \\ \hline
			\textbf{Test environment} & Sony Ericsson Xperia X10 running Android 2.1.1.A.0.6 with kernel: 2.6.29 \\ \hline
			\textbf{Pre-conditions} & Test 2 completed \\ \hline
			\textbf{Post-conditions} & App is in same state as before test started \\ \hline
			\textbf{Result} & PASS \\ \hline
		\end{tabular}
		\caption{Summary of test 6}
	\end{figure}

	\begin{figure}[htb]
		\centering
		\begin{tabular}{|p{5.0cm}|p{5.0cm}|p{1cm}|}
			\hline \textbf{Action} & \textbf{Expected outcome} & \textbf{Result} \\ \hline
				1. Tap "Lagrede Observasjoner" from the main view &
				A list containing one observatin (from test 2) is displayed &
				PASS \\ \hline

				2. Select observation 1 from the list &
				The observation from test 2 is displayed in the same state as
				earlier (two species) &
				PASS \\ \hline

				3. Tap "Detailer" in the row containing "grågås" &
				Details view for "grågås" observation is displayed, the field "Aktivitet"
				is filled in with "Rugende" &
				PASS \\ \hline
		\end{tabular}
		\caption{Execution of test 6}
	\end{figure}

\newpage
\subsection*{Test 7 (run 3)}

	\begin{figure}[htb]
		\centering
		\begin{tabular}{|p{3.5cm}|p{7.0cm}|} \hline
			\textbf{Requirement} & F7 \\ \hline
			\textbf{Version} & 1.0 \\ \hline
			\textbf{Date} & 2011-11-08 \\ \hline
			\textbf{Tested by} & Stian Liknes \\ \hline
			\textbf{Test environment} & Sony Ericsson Xperia X10 running Android 2.1.1.A.0.6 with kernel: 2.6.29 \\ \hline
			\textbf{Pre-conditions} & Test 6 completed, still viewing stored observation \\ \hline
			\textbf{Post-conditions} & Observation is stored on device with an additional row containing ("Art" = "grønnfink", "Antall" = 9) \\ \hline
			\textbf{Result} & PASS \\ \hline
		\end{tabular}
		\caption{Summary of test 7}
	\end{figure}

	\begin{figure}[htb]
		\centering
		\begin{tabular}{|p{5.0cm}|p{5.0cm}|p{1cm}|}
			\hline \textbf{Action} & \textbf{Expected outcome} & \textbf{Result} \\ \hline

			1. Add a new row with the same procedure as in test 2 &
			A empty row is appended to the observation &
			PASS \\ \hline

			2. Fill in ("Art" = "grønnfink" and "Antall" = 9) in the new row
			using the same procedure as in test 2. &
			New row is populated with ("grønnfink", 9) &
			PASS \\ \hline

			3. Tap "Lagre" &
			Observation stored with an additional row ("grønnfink", 9) &
			PASS \\ \hline

		\end{tabular}
		\caption{Execution of test 7}
	\end{figure}

\newpage
\subsection*{Test 8 (run 3)}

	\begin{figure}[htb]
		\centering
		\begin{tabular}{|p{3.5cm}|p{7.0cm}|} \hline
			\textbf{Requirement} & F9 \\ \hline
			\textbf{Version} & 1.0 \\ \hline
			\textbf{Date} & 2011-11-08 \\ \hline
			\textbf{Tested by} & Stian Liknes \\ \hline
			\textbf{Test environment} & Sony Ericsson Xperia X10 running Android 2.1.1.A.0.6 with kernel: 2.6.29 \\ \hline
			\textbf{Pre-conditions} & Test 1 completed, still viewing stored observation \\ \hline
			\textbf{Post-conditions} & App is in same state as before test execution \\ \hline
			\textbf{Result} & PASS \\ \hline
		\end{tabular}
		\caption{Summary of test 8}
	\end{figure}

	\begin{figure}[htb]
		\centering
		\begin{tabular}{|p{5.0cm}|p{5.0cm}|p{1cm}|}
			\hline \textbf{Action} & \textbf{Expected outcome} & \textbf{Result} \\ \hline
			
			1. Tap "GPS" &
			Longitude and latitude contains coordinates near the current location &
			PASS \\ \hline

		\end{tabular}
		\caption{Execution of test 8}
	\end{figure}

\newpage
