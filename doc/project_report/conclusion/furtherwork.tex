\subsection{Conclusions}
The goal of this project has been to teach us software engineering and teamwork skills in the context of a development project to make a realistic prototype of an information system for a real world customer. Overall we feel the project has been quite successful in this regard.

We feel the project has been helpful in preparing us for a real software development environment. We had very few problems in our communication with our customer, and in the end we left our customer satisfied with our work. Particularly our pre-study and research into the various cross-compiling frameworks that were available.

The course also provided valuable experiences in group dynamics. Our group has been quite diverse in both culture and experience, which has required us to deal with issues like role allocation, work load management, and other group management skills. We have had very few internal conflicts in our group, something that no doubt has been a big advantage.

%We dealt with a number of new and challenging aspects of software development. It, indeed, was a unique course well designed to provide us with the opportunity to work in completely different environment than the usual school exercises we are normally used to. From working seamlessly with people of different cultural background to our coordinated effort in selecting the right implementation technology to making our customer satisfied to fulfilling our requirements, we have been able to gain an invaluable insight on how to communicate, schedule, execute and deliver expectations under pressure. We have had challenges, shortcomings and unexpected events that were out of our control but the level of diligence the member of the groups showed and commitment for the successful completion of the project are some of the things that are worth mentioning.

The result of our work is a fully functional prototype. We managed to fulfill all but one of our functional requirements. However, our main non-functional problem was that our app would be able to compete with the previous best method of observing species: using a pen and notebook in the field, and then manually entering the information into the artsobservasjoner.no website. 

Unfortunately, the frameworks that allowed us to quickly develop a cross platform app (PhoneGap, JQuery Mobile) also made our app quite sluggish at times. However, our app offers several advantages over the notebook, such as auto-complete of species names, simple entry of dates and times, geolocation data and built in picture functionality. The data is also exported in a format that is easily imported to the website, significantly reducing the amount of time needed in that part of the observation process. 

In the end, we are convinced that with some additional work on optimizing our frameworks and perhaps with a future possibility of uploading observations directly or otherwise streamlining the export/import process, our app could substantially ease and speed up the process of species observation.

%Our implementation went according to plan and our use of JavaScript in PhoneGap was quite effective.
%PhoneGap had its limitations and reduced performance compared to native implementations but helped us with GUI work and allowed us to code in an environment and language we were already familiar with.
%The app is a fully functional prototype, but will probably need a little tweaking before fully releasing it to the public.
%Our customer also found our research and issues with the usability of the app interesting for their further work with mobile implementations of observations.
%Mainly how convenient it actually is, undeniably a large portion of the users will not think it is faster or easier to use it over a sheet of paper.
