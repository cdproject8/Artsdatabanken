\appendix

\section{Project directive and templates}
	
	\subsection{Contact information}
		\subsubsection{Customer}
			\begin{itemize}
				\item {\bf Nils Valland} \newline
						Telephone: 73 59 23 01 \newline
						Mobile: 92 41 20 37 \newline
						Fax: 73 59 22 40 \newline
						E-mail: nils.valland@artsdatabanken.no
				\item {\bf Askild Olsen} \newline
						Telephone: 73 59 21 93 \newline
						Mobile: 91 78 34 89 \newline
						Fax: 73 59 22 40 \newline
						E-mail: askild.olsen@artsdatabanken.no
			\end{itemize}
			
		\subsubsection{Supervisor}
			\begin{itemize}
				\item {\bf Muhammad Asif} \newline
						Telephone: 73 59 36 71 \newline
						E-mail: muhamma@idi.ntnu.no
			\end{itemize}

		\subsubsection{Team members}
				\begin{itemize}
					\item {\bf Anders Søbstad Rye} \newline
							E-mail: anderrye@stud.ntnu.no
					\item {\bf Andreas Berg Skomedal} \newline
							E-mail: andrskom@stud.ntnu.no
					\item {\bf Dag-Inge Aas} \newline
							E-mail: dagingaa@stud.ntnu.no
					\item {\bf Muhsin Gnaydin} \newline
							E-mail: gunaydin@stud.ntnu.no
					\item {\bf Nikola Djoric} \newline
							E-mail: nikoladj@stud.ntnu.no
					\item {\bf Stian Liknes} \newline
							E-mail: stianlik@stud.ntnu.no
					\item {\bf Yonathan Redda} \newline
							E-mail: redda@stud.ntnu.no
				\end{itemize}
	\newpage
	\subsection{Meeting agendas}
		\begin{figure}[htb]
			\centering
			\includegraphics[width=0.8\textwidth]{appendix/meeting_agenda.jpg}
			\caption{Meeting agenda}
			\label{fig:meeting-agenda}
		\end{figure}
	
	\newpage
	\subsection{Meeting minutes}
		\begin{figure}[htb]
			\centering
			\includegraphics[width=0.8\textwidth]{appendix/meeting_minutes.jpg}
			\caption{Meeting minutes}
			\label{fig:meeting-minutes}
		\end{figure}
	
	\newpage
	\subsection{Weekly status report}
		\begin{figure}[htb]
			\centering
			\includegraphics[width=0.8\textwidth]{appendix/weekly_status_report.jpg}
			\caption{Weekly status report}
			\label{fig:weekly-status-report}
		\end{figure}

\newpage
\section{User guide TODO}
This portion of the document details installation guides for the user, in addition to a how-to for the application. This should provide the user with adequate information about how to install and use the application.
\subsection{Installation guide}
The application can be downloaded from the Android Market, with the name "Artsdatabanken". It can also be installed directly from an executable installation package, an apk, by going to the following url: \url{http://stuff.daginge.com/artsdatabanken.apk}.

However, if the user wishes to install the latest build, see Developers guide.

\subsection{How-to}
\subsubsection{Creating an observation}
The user creates an observation by following the "Create and observation/Ny Observasjon" link from the front page of the application.
From here, the user selects which species group to observe. 
This will lead the user to an observation table, where the user can input species name, which is auto-completed, and the number of individuals found. 
Both common names and scientific names are applicable. 

\subsubsection{Add additional information}
From the observation page, the user can choose to add more information about the species observation by selecting "Add more information/Detaljer".

All fields are available for editing, simple to use boxes will open if date or time fields are edited.

\subsubsection{Add pictures of a species}
At the bottom of the Extended information page the user may also include images from disk or camera, if the user clicks on an already existing image he/she can remove the picture from the observation again.

\subsubsection{Adding additional species to an observation}
The user may also add additional species to an observation via the "Add additional species/Legg til ny art" button. This will create a new row for that species.

\subsubsection{Storing an observation}
To store an observation simply click the "Save/Lagre" button at the bottom of the observation.

\subsubsection{Editing a stored observation}
To edit an existing observation the user can select "Saved observations/Lagrede Observasjoner" from the front page. 
This will list all saved observations on the device. 
These observations are identifiable from it's species type, date of creation and an id assigned to it.

\subsubsection{Exporting an observation}
To export an observation the user can click on the "Export/Eksporter" button at the bottom of the observation page.
This will open up a menu to select which application to use for exporting, the recommended choice is an email client.
The user is free to send the observation to any email address, the received email can be used on the web-page of Artsdatabanken by copying the entire email body and pasting it in the import from XSL section.
Any images linked to the observation will be attached to the email.

\subsubsection{Deleting an observation}
An observation can be deleted by clicking the "Delete/Slett" button at bottom of an observation.

\section{Developers guide}

This section will in details help maintaining developers understand and use the
code we have produced during the course of this project. It will also include
comments about the ideas we have for the further development of this app.

\subsection{Further work}
At this time this is a simple application for creating and submitting
observations. Improvements in design and functionality shouldn't be an issue with the current framework,
and the potential is present. First priorities might be the incorporation of a
direct submission API towards Artsdatabankens services, to avoid having to
export through mail and import again. 

The implementation of our final functional requirement that involves being able to update
the internal database of autocomplete names is also high on the list. In order for this to be
less demanding on the phone and less of an implementation challenge it is strongly suggested 
that Artsdatabanken supplies a refined API for this.
Meaning one that offers the exact data needed so no reparsing of large XML files is needed locally.

Optimizing screen use might also be an
idea, at the time objects are fairly large and space consuming, but this is
more of a design philosophy from JQuery Mobile and is debatable.

The current version is also largely focused on the interface prefered by birdobservers, 
while trying to be universial it might be neccessary to further customize the different 
layouts or stored fields for observations of different species groups.

Adding a section of the application for viewing species and obtain more
information about them was a subject from the beginning of the project.  But it
was of less importance and quickly deemed outside the scope of the original
application.

At the current time most object methods are implemented functionally and not by
using the "prototype" construct of JavaScript.  At this point there isn't many
concurrent objects in use at the same time so it shouldn't be much of an issue.
However as the application progresses this should most likely be changed to
improve performance.

\subsubsection{Use of API}

We used Artsdatabankens webtjenesterbeta API to download species names for the
auto-complete. By inputing species-categories to the API we were able to
download lists of species names, these were returned in an XML format. To
simplify the app (and save storage space), we parsed the XML into JSON and
shipped the data as a part of the app. We used the following XSLT-transformation
to parse the data:

\begin{lstlisting}
	<xsl:template match="/">
		<xsl:apply-templates 
			select="//adb:scientificName | 
			//adb:vernacularName" 
	/>
	</xsl:template>

	<xsl:template 
		match="//adb:scientificName | 
		//adb:vernacularName"
	>'<xsl:value-of select="." />',
	</xsl:template>
\end{lstlisting}

An obvious extension is to parse using javascript in the app itself, so
that users can "sync" the species listings at their own convenience. It is a
fairly simple transformation that can be implemented using jQuery-selectors.
Currently we store one file per category, each file contains a function named
"autocompleteData" that returns a list of names. These files could be updated
using PhoneGap's File API.

\subsubsection{Code repository}

The code and documentation from this project can be found in its entirety at
the following url: \url{https://github.com/cdproject8/Artsdatabanken}. Included
here is a complete revision history for our entire project, all of our
documentation excluding meeting minutes and agendas, in addition to the app
itself. The application is ready to compile from this source.

For further work, we recommend you fork this project. The project is licensed
under Creative Commons Attribution-ShareAlike 3.0 Unported, and all further
work should also be licensed under the same or similar licenses.

\subsubsection{Maintenance}
A short text about maintenance, and the challenges ahead.
