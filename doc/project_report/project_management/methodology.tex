\subsection{Methodology}

For this project, the team decided to use an adoption of the SCRUM methodology, an agile methodology. The other option was to use the waterfall model. After the preliminary study, it is considered inappropriate, as it requires a very strict definition of user requirements before the beginning of the development process. It is expected that the user requirements may change significantly during the development of the application, and in the beginning, the project and its domain are still very loosely defined. In addition, the customer and the advisor from IDI recommended use of an agile methodology. More about choice of development process in section 'Preliminary study'.

Planned work will consist of daily stand-up meetings, where all group members will be present and discuss work completed during the previous day and plans for the following day. 
In addition to this, main internal group meeting will be held on Mondays, and meetings with the advisor and the customer will mostly be scheduled for Tuesdays.
Each sprint will be planned in internal meetings and the end of sprint results will be presented to both customer and advisor.
The sprint plan will also be agile, taking into consideration the ever changing requirements. 
The team will have a product owner and a scrum master, but will not strictly employ standards such as a SCRUM board. 

\subsection{Project phases}

Our project will loosely follow the plan in Figure~\ref{gantt:project}.

\begin{figure}[h]
\centering
  \includegraphics[width=1.0\textwidth]{project_management/project_effort_estimation}
  \caption[Gantt chart of project phases]{Gantt chart showing project phases, and when they are planned done.}
  \label{gantt:project}
\end{figure}
First week for this project is planed as an introductory week, where team members will get introduced to each other and get familiar with the project and following assignments.
After that, the first official phase of the project will last for two weeks, and refers to planning and project management. 
During this period all administrative tasks should be finished, allowing a smooth start of the actual developing process. 
At the same time, the preliminary study will be one of the occupations for all team members. 
This should give a clear picture of available technologies, development processes and project work flow for the upcoming months.
The start of the actual development is scheduled for week 36, overlapping with the previous phase. 
The development process will consist of our "sprint zero", lasting one week, and four two-week sprints. 
At the end of sprint four, in week 44, the team should finish the application development and provide a release version to the customer.
The last three weeks are reserved for finishing the documentation, which was written throughout the whole previous phases. 
During this period, the remaining sections should be finished, the whole document thoroughly revised and prepared for the final delivery.

\subsubsection{Planning and research}

This first phase of the project consists of an introduction to the course,
planning of the project, introduction to the problem domain, getting to know
the group, requirement gathering and such. In this phase the most important
decisions about system architecture, choice of COTS and framework, project
methodology and role distribution, will be taken. Figure~\ref{gantt:pre_imp}
shows how the work will be distributed in this time period.

\begin{figure}[h]
\centering
  \includegraphics[width=1.0\textwidth]{project_management/pre_implementation_gantt}
  \caption[Gantt chart of planning and research phase]{Gantt diagram picturing how work should be distributed in the time available during the planning and research phase.}
  \label{gantt:pre_imp}
\end{figure}

\subsubsection{Sprints}

Each sprint will consist of four phases. Effort estimation is
detailed in Table \ref{Sprint effort estimation}.

\begin{table}[htbp]
\begin{center}
  \begin{tabular}{|r|r|r|}
    \hline
    \bf{Task} & \bf{Hours per person} & \bf{Hours total} \\
    \hline
    Planning & 8 & 56 \\
    Implementation & 8 & 56 \\
    Testing & 8 & 56 \\
    Documentation & 16 & 112 \\
    Administrative & 8 & 56 \\
    \hline \hline
    \bf{Sum} & 50 & 350 \\
    \hline
  \end{tabular}
  \caption{Task effort estimation for each sprint}
  \label{Sprint effort estimation}
\end{center}
\end{table}

\textbf{Planning} The planning phase of each sprint represents the time
required for work distribution, planning of how each task should be handled,
and how testing should be done for this section.

\textbf{Implementation} The implementation phase represents time spent on coding.
This will also include code refactoring and other maintenance tasks
related to the code.

\textbf{Testing} The testing phase represents time spent testing the system.
This includes integration testing, unit testing, functional testing etc.
The testing and implementation phases will work concurrently due to the
test-driven development methodology.

\textbf{Documentation} The documentation phase represents time spent
documenting work effort (implementation, research, etc.) and administrative
tasks like meetings.

\subsubsection{Documentation}
The last part of the project will consist of evaluating our work, and finish the documentation of the project. In addition, this part will include a final presentation November 24th for the external examiner, advisor and customer. This period will span the last three weeks of the project.  
