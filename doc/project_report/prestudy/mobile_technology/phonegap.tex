\paragraph{\bf{Phonegap}}
PhoneGap is an HTML5 app platform that allows you to author native applications with web technologies and get access to APIs and app stores. PhoneGap leverages web technologies such as HTML and JavaScript. PhoneGap is the only app platform available today that can publish to 6 platforms.
\cite{phonegap:about}
	\begin{itemize}
		\item Applications can be developed for Apple’s iOS, Google’s Android,
		Microsoft’s Windows Mobile, Nokia’s Symbian OS, RIM’s BlackBerry and Bada.
		\item Enables developers to take advantage of JavaScript, HTML5 and CSS3,
		which they might have already been familiar with.
		\item Access Native features such as compass, camera, network, media,
		notifications, sound, vibrate and storage etc.
    \item Can use existing CSS and Javascript libraries directly in your code
    \item Seems like a native application when in reality it's an offline web application
    \item Provides a build tool for automatically building binary application packages for six different platforms
    \item Provided under the (new) BSD license or alternativly the MIT licence, the framework is entierly Open Source and free for Open Source projects.
    \item Provides a well-written API, geared towards web developers
	\end{itemize}

\paragraph{Corona}
Corona enables to develop graphically rich multimedia applications based on a programming
language Lua\cite{corona:about}. Corona supports both the iOS and Android platforms. Corona focuses on
applications with a lot of graphical animation. Corona is neither open source nor free, and also currently a developer
has to pay \$99 yearly to maintain membership and be able to build applications for App Store.

\paragraph{Titanium}
Titanium is a cross-platform mobile applications development framework that uses web
development technologies to develop applications for the Android and iOS mobile platforms.
Titanium is an open source framework that uses JavaScript and JSON as application language;
it can also use Python, Ruby and PHP scripts\cite{titanium:about}
