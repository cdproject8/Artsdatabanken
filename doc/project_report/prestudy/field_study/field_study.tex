\subsection{Field study}
\label{sec:field-study}
The field study was conducted the 5th of October at Artsdatabankens headquarters
in Trondheim. The following are our findings after an example species
observation and a talk about species observations in general.

Species are registered individually or in collaboration with two or
more individual observers. Observations or Sightings are a collection of species
and attributes about the species such as name, the number of observed
individuals, the activity they were engaged in during observation, age, sex,
observation start-end with date and time, observer, co-observer, and location.
Registered data quality is validated by reputation and bibliometric qualities.

\begin{figure}[htb]
	\centering
	\includegraphics[width=0.8\textwidth]{prestudy/field_study/field_stud.jpg}
	\caption{Counting number of different species}
	\label{fig:field_study}
\end{figure}

Each observations must have a sightings location. Right now the user has to
register the location manually while submitting the observation online.
Sighting location and species picture are non-negotiable features of the system.
Location can be known and already registered or can be conjured up using GPS
coordinates. Locations are classified as super and sub-locations. Pictures are
used based on convenience such as only for plants and immobile species candidates.

The customer wants the team to provide full mobility to the observers with
offline data storage and synchronization, and a thorough comparison of the iOS
and Android platform possibilities together with what they stand to gain or lose in
adopting the technologies. The customer also suggested keeping all the
functionalities of the desktop, and breaking down those functionalities on the
mobile device.

During any sighting, the recommendation is to use approximate number of species
seen for common species, while being strictly accurate for rare species or not
mentioning a number. Each observation is limited to one type of species, for
example birds, plants or mammals. There will not be any mixing of a number of species for a single observation.

\begin{figure}[htb]
	\centering
	\includegraphics[width=1\textwidth]{prestudy/field_study/flora_fauna_Nikola.jpg}
	\caption{Observation can include rare and endangered species}
	\label{fig:field_study_species}
\end{figure}


