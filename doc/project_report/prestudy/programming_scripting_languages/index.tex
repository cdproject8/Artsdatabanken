\subsection{Mobile Development}
It is only natural that developers have a taste for different languages or
learn a new one for as a project in software development. Most of the time,
scripting languages are used to develop web applicaitons. Mobile applications
can also be developed using one or a combination of C, C++, Java, or a whole raft of scripting languages such as HTML, CSS or JavaScript.

\subsubsection{Native languages}
Most mobile platforms come with their own preferred programming language. iOS is developed in Objective-C and Android in Java. This requires developers to learn a new language when creating applications for new mobile platforms.

\subsubsection{HTML5, CSS3 and JavaScript}
An alternative way to app development is using making the app live inside the browser of the phone. These webapps are developed using existing web standards used by developers all over the world. Recently, phone browsers have become more complex, enabling the developers to code applications more powerful than ever before. However, webapps are much slower than running native applications, but can be deployed to any phone with a sufficiently complex browser.

\subsubsection{jQuery Mobile}
This is a cross-platform and cross-device framework with which one can write
a mobile application capable of running on anyone of the popular mobile platforms.
jQuery Mobile is fit for developing applications with touch input, requiring
less processing power and  makes this possible through a lightweight code
built with progressive enhancement and flexibility in mind.

In addition, our customer has previous experience using this framework, and has provided us with a custom built CSS designed for Artsdatabanken. This will enable us to easily conform to design guidelines set by Artsdatabanken.

\subsubsection{Conclusion}
While programming native applications creates a faster, more user friendly experience, the domain knowledge necessary to test and deliver these applications would take a lot of effort in our group, which would mean we would spend more time learning new programming languages than fulfilling the requirement specification. However, everyone in the group has little to a lot of knowledge making in applications using HTML5, CSS3 and javascript. We chose a development framework based on this previous experience.

Additionally, the customer's requirement was to build an application that is capable of running on multiple platforms. PhoneGap currently supports up to 6 platforms in addition to being open source and using a well known web technology to author the applications and later compile them into native applications. PhoneGap has also been getting recognitions by the likes of established technology companies such as Adobe systems.  Adobe systems has officially integrated PhoneGap in its 5.5 version of Creative suite. While Titanium is a cross platform framework, it only supports Android and iOS. Corona supports only Android and iOS, it is not free and its focus is heavy multimedia content and graphics.

