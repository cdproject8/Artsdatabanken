\subsection{Development process}

\subsubsection{SCRUM model}
	
The most used model for iterative, incremental development of projects today
is SCRUM\cite{scrumi}. It is an agile software development strategy which uses iterative
development as a basis, but advocates a lighter and more people-centric viewpoint
than traditional approaches. Agile processes use feedback, rather than planning
as their primary control mechanism. The feedback is driven by regular tests and
releases of the evolving software\cite{wiki:development-process}.\newline
	
The Scrum approach was originally suggested for managing product development
projects, but its use can also be focused on the management of software development projects.\newline
	
Scrum is a process skeleton that contains sets of practices and predefined roles.
The main roles in Scrum are:	
\begin{itemize}
	\item the ScrumMaster, who maintains the processes (typically in lieu of a project manager)
	\item the Product Owner, who represents the stakeholders and the business
	\item the Team, a cross-functional group who do the actual analysis, design, implementation, testing, etc.
\end{itemize}

\begin{figure}[htb]
	\centering
	\includegraphics[width=0.8\textwidth]{prestudy/development_process/scrum.png}
	\caption{Scrum methodology\cite{targetprocess:scrum}}
	\label{fig:scrum-methology}
\end{figure}

During each sprint, typically a two to four week period (with the length decided by the team),
the team creates a potentially deliverable product in each increment.
Features that go into a sprint come from the product backlog,
which represents a prioritized set of high level requirements of work that should be done.
Which backlog items are going into the sprint is determined during the sprint planning meeting.
Product Owner is responsible to inform the team of the items in the product backlog that should be completed.
The team then determines how much of the requirements they can complete during the next sprint,
and records this in the sprint backlog.
When sprint starts, no one is allowed to change the sprint backlog, which means that the requirements
can not be modified for that sprint. Each sprint must end on time, and development is timeboxed.
If any of the requirements are not completed for any reason, they are left out and returned to the product backlog.
When each sprint is completed, the team demonstrates what have been done, and how to use the software.

Scrum is very convenient because it enables the creation of self-organizing teams by encouraging verbal communication between all team members situated at one location.

Most important principle of Scrum is its that during a project the customer can change their
mind about what they want and need, and that unpredicted challenges cannot be easily addressed
in a traditional predictive or planned manner.
As such, Scrum adopts an empirical approach, accepting that the problem cannot be fully understood
or defined at the start, focusing instead on maximizing the team’s ability to deliver quickly and
respond to new requirements.\cite{wiki:development-process}

\subsubsection{Waterfall model}
The waterfall model is a sequential design process in which progress is seen as flowing steadily downwards through the predefined phases, like a waterfall. Phases are Conception, Initiation, Analysis, Design, Construction, Testing, Production/Implementation and Maintenance, and each this phases is executed sequentially. This model is one of the most used software development process this days, beside SCRUM. 

\begin{figure}[htb]
	\centering
	\includegraphics[width=0.8\textwidth]{prestudy/development_process/waterfall.jpg}
	\caption{The unmodified "waterfall model" methodology\cite{worldpress:waterfall}}
	\label{fig:waterfall-model}
\end{figure}

Original waterfall model is called Royce's model, and it defines following phases:

\begin{itemize}
	\item Requirements specification
	\item Design
	\item Construction (i.e. implementation or coding) testing, etc.
	\item Integration
	\item Testing and debugging (i.e. Validation)
	\item Installation
	\item Maintenance
\end{itemize}

Specific about waterfall model is that each phase is started after previous phase is finished, and there is no overlapping between 2 phases. This applies for most strict model, but there are also various modified models that may include slight or major variations upon this process.

\subsubsection{Conclusions}
For this project, the team decided to use the SCRUM model. We chose this method because of the highly volatile nature
of requirements. The project description and domain is very loosely defined,
and we expect the requirements to change significantly during the development
of the application. But as SCRUM requires strict and precise
execution of the predefined model, the actual development process will be slightly modified.
The project will have 4 sprints, including an additional 'zero' sprint which contained the initial
prestudy and GUI sample. Each sprint will span 2 weeks, or 10 working days.
Some additional time will be left after completion of the sprints for revision,
fixing and more testing, to improve the product's reliability and quality if necessary.
