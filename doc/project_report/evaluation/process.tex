\subsection{Development process, methodology and Workflow}

We started on this project as part of a course work for TDT4290---Customer Driven Project. In this project, we have been able to walk through the processes of software development efforts an average software engineer would have gone through in an industrial setting. We have tried to keep the feel and the work as standard as possible and replicated software development processes as properly as we can. This gave us an invaluable insight into how to handle, schedule, manage, trace and carry out tasks in software development projects and below is an evaluation of our experience in this project.
	\subsubsection{Development methodology---Scrum}

With scrum being the most popular software development methodology these days plus all the team members have had some sort of experience with scrum which helped us save a considerable amount of time we would have spent learning the basics of scrum and that, among other things, scum enables to break down tasks into several manageable chunks made scrum an ideal methodology to employ in our project as well. Adopting pure scrum seemed rather less optimistic since all team members had responsibilities in other courses with demanding projects and attending five days of stand up meeting consistently by all groups looked unrealistic. Scrum is also complex administrative wise and team members might be lost in all the updating and stand-up meetings. And since we started this project in order to develop a prototype, scrum made it easy to revise requirements, implementations and features in constant and periodical contact with our customer. To make sure such things went smoothly, we had to get together early and decide on how we follow scrum and made amends that suited all members. we decided to hold a stand-up meeting---a time slot when everyone will be able to inform other members of self-completed task and plan for the day ahead---to keep abreast of changes in the the development project, keep sprint lengths to maximum of two weeks. This arrangement has been so effective we had negligible problems. One of the challenging thing to do associated with our scrum was to get something done on each day, to wake up early and show up every Monday to Thursday morning at 9 to relate our work. Some group members work late into the night and making it early in the morning was a bit of an inconvenience but we decided to stick to the plan and that has helped us complete a greater portion of requirements gathering and analysis task early for approval with our customer. We did not employ a post it note row as part of our sprint work backlog, we depended on oral communication and everyone took the responsibility of completing the missing parts depending on the stand-up meetings every morning. At times, we had to explicitly assign tasks if we felt that some unpopular tasks were being bounced around. In general, the flexible division of work, the consistent stand-up meeting, and the minimum threshold we set for asking questions together with the team members familiarity with scrum has made our work relatively trouble free.
	\subsubsection{Implementation}
	\subsubsection{Group dynamics}
	\subsubsection{Time estimation}
	\subsubsection{Work load}
	\subsubsection{Seminars and study process}


