\subsection{Developers guide}

This section will in details help maintaining developers understand and use the
code we have produced during the course of this project. It will also include
comments about the ideas we have for the further development of this app.

\subsubsection{Further work}
At this time this is a simple application for creating and submitting
observations. Improvements in design and functionality shouldn't be an issue with the current framework,
and the potential is present. First priorities might be the incorporation of a
direct submission API towards Artsdatabankens services, to avoid having to
export through mail and import again. 

Optimizing screen use might also be an
idea, at the time objects are fairly large and space consuming, but this is
more of a design philosophy from JQuery Mobile and is debatable.

The current version is also largely focused on the interface prefered by birdobservers, 
while trying to be universial it might be neccessary to further customize the different 
layouts or stored fields for observations of different species groups.

Adding a section of the application for viewing species and obtain more
information about them was a subject from the beginning of the project.  But it
was of less importance and quickly deemed outside the scope of the original
application.

At the current time most object methods are implemented functionally and not by
using the "prototype" construct of JavaScript.  At this point there isn't many
concurrent objects in use at the same time so it shouldn't be much of an issue.
However as the application progresses this should most likely be changed to
improve performance.

\paragraph{Use of API}\hspace{1mm}\newline

We used Artsdatabankens webtjenesterbeta API to download species names for the
auto-complete. By inputing species-categories to the API we were able to
download lists of species names, these were returned in an XML format. To
simplify the app (and save storage space), we parsed the XML into JSON and
shipped the data as a part of the app. We used the following XSLT-transformation
to parse the data:

\begin{lstlisting}
	<xsl:template match="/">
		<xsl:apply-templates 
			select="//adb:scientificName | 
			//adb:vernacularName" 
	/>
	</xsl:template>

	<xsl:template 
		match="//adb:scientificName | 
		//adb:vernacularName"
	>'<xsl:value-of select="." />',
	</xsl:template>
\end{lstlisting}

The implementation of our final functional requirement that involves being able to update
the internal database of autocomplete names is also high on the list of further work.
So that users can "sync" the species listings at their own convenience.
In order for this to be less demanding on the phone and less of an implementation challenge it is strongly suggested that Artsdatabanken supplies a refined API for this.
Meaning one that offers the exact data needed so no reparsing of large XML files is needed locally.

If not, an obvious extension is to parse using javascript in the app itself. It is a
fairly simple transformation that can be implemented using jQuery-selectors.
Currently we store one file per category, each file contains a function named
"autocompleteData" that returns a list of names.
These files could be updated using PhoneGap's File API.

\paragraph{Code repository}\hspace{1mm}\newline

The code and documentation from this project can be found in its entirety at
the following url: \url{https://github.com/cdproject8/Artsdatabanken}. Included
here is a complete revision history for our entire project, all of our
documentation excluding meeting minutes and agendas, in addition to the app
itself. The application is ready to compile from this source.

For further work, we recommend you fork this project. The project is licensed
under Creative Commons Attribution-ShareAlike 3.0 Unported, and all further
work should also be licensed under the same or similar licenses.

\paragraph{Maintenance}\hspace{1mm}\newline

A short text about maintenance, and the challenges ahead.
