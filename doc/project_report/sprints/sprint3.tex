\section{Sprint 3}

\subsection{Sprint Planning}

In this sprint, the team worked on rectifying the technical errors in the
implementation of sprint two requirements. More work was required to finish the
auto-complete, removing names that are not species. Being able to add new
species when editing an observation and fixing storage on the phone was one of
the modifications required.

\subsubsection{Expected results}

Getting semi completed local storage and auto-complete functionalities to be
completed, deciding how to best export locally stored observations, and team is
expected to divert its resource more getting the system to work on the android
platform, discussing the terms of usability testing with the customer.

\subsubsection{Duration}

Sprint 3 will have a duration of two weeks between October 10 (beginning of
week 41) to October 23(end of week 42). The customer meeting was held on
Wednesday October 12, 2011 and the advisor meeting was on October 11 and
October October 18.

\subsection{Requirements}

Some of the requirements are from sprint 2 and the goal is to make them free
from bugs and issues and focus more on exporting the observation.

\begin{description}
	\item[F4] User must be able to export stored observation so they can be uploaded to Artsdatabanken later
	\item[F7] User must be able to edit a stored observation on the device
\end{description}

\subsection{Implementation}

Editing of observations was undertaken in the previous sprint as this was a very simple and quick extention of the view observations functional requirement.

\subsubsection{Storage}

\label{sprint3:storage}

We had an issue with Storage not working correctly due to a browser
implementation on newer versions of Android. This lead to the application
not working on many  phones. In the end, the problem was that Android
version greater than 2.1 does not support null-callbacks in storage functions, which
lead to a fatal error while creating an observation.

	\subsubsection{Export}

	The main target of this sprints implementation was to export the
	observations in a format readable by Artsdatabasen 2.0. This was achieved
	using a format similar to csv, but with tab separated values instead of
	commas. The data is sent in clear text to an email specified by the user,
	where the user can easily copy paste the data into a form on Artsdatabanken
	system. A PhoneGap plugin, WebIntent \cite{library:webintent}, was used for triggering the email
	event on Android phones. No solution for iPhone devices has been found,
	however we have observed that several solutions are available.
	
	Our export system is based on a simple solution where all the fields of a species
	is retrieved by the observation object for each of it's species, printed to a string
	and concatenated.
	
	This string is then sendt via the WebIntent plugin which sends a JSON object from
	JavaScript to the plugin which is written in Java and runs on the android platform.
	the observation string is sent as metadata with a "SEND" intent which will promt 
	the user for an appropriate program to handle it. Typically an email client as
	suggested by the customer.

\label{Auto-complete}

By testing on mobile devices we have ruled out some performance issues and bugs,
at this point the auto-complete is usable.

\subsection{Testing}

Previously we had done most of our testing on using PCs, in this sprint we
focused more on mobile devices. We uncovered some problems with the
storage functionality (see \ref{sprint3:storage}). Additionally we discovered
a bug in the auto-complete that caused it to load excessive amounts of data from
the file system.

\subsection{Customer Feedback}

The customer was satisfied with our work and the progress. The customer
also stressed that the team make the export system work, and suggested that
the team also implement things in less complex and practical manner. The
customer considers the exporting functionality a deal breaker or maker of
the prototype and the team was advised to complete that as soon as possible
according to the schedule and expunge existing errors in the implementation
so far.

\subsection{Evaluation}

The sprint generally went well but exporting required a lot of time and
resource on the programming department, and later the team had to put it on
sprint 4 backlog in front for further work. Sprint 3 has seen about eighty
percent completion of the total requirement, the team's progress is considered
acceptable by both the advisor and the customer. 
