\section{Sprint 3 TODO}

	\subsection{Sprint Planning}
In this sprint, the team worked on rectifying the technical errors in the implementation of sprint two requirements. More work was required to finish the auto-complete, removing names that are not species. Being able to add new species when editing an observation and fixing storage on the phone was one of the modifications required.
	\subsubsection{Expected results}
Getting semi completed local storage and auto-complete functionalities to be completed, deciding how to best export locally stored observations, and team is expected to divert its resource more getting the system to work on the android platform, discussing the terms of usability testing with the customer.
	\subsubsection{Duration}
Sprint 3 will have a duration of two weeks between October 10 (beginning of week 41) to October 23(end of week 42). The customer meeting was held on Wednesday October 12, 2011 and the advisor meeting was on October 11 and October October 18.
	\subsection{Requirements}
Some of the requirements are from sprint 2 and the goal is to make them free from bugs and issues and focus more on exporting the the observation.
\begin{description}

\item[F3] User must be able to add more species to an observation

\item[F4] User must be able to export stored observation so they can be uploaded to Artsdatabanken later

\item[F7] User must be able to edit a stored observation on the device
\end{description}

	\subsection{Implementation}

	\subsection{Testing}

	\subsection{Customer Feedback}
The customer was satisfied with our work and the progress. The customer also stressed that the team make the export system work, and suggested that the team also implement things in less complex and practical manner. The customer considers the exporting functionality a deal breaker or maker of the prototype and the team was advised to complete that as soon as possible according to the schedule and expunge existing errors in the implementation so far.
	\subsection{Evaluation}
The sprint generally went well but exporting required a lot of time and resource on the programming department, and later the team had to put it on sprint 4 backlog in front for further work. Sprint 3 has seen about eighty percent completion of the total requirement, the team's progress is considered acceptable by both the advisor and the customer. 