\section{Sprint 0}

\subsection{Sprint Planning}
For this sprint, we plan to get familiar with the project, and make important
decisions about technologies and development processes. We will conduct a field
study at Artsdatabanken where we will get an introduction into species
observation and the process behind it. In addition to this, we will make a
prototype of the user interface in jQuery Mobile to show the customer what our
plan for the application is. This is done to clear up most of the
misunderstandings about the requirements as early as possible, and give the
customer feedback about what our vision of the application is. 

Sprint 0 is an introductionary sprint, and will not run for two full weeks.

\subsubsection{Expected results}
This sprint should provide significant
insight into the problem domain, and how we are going to solve it. The
preliminary study is a very important part of this sprint. In addition, the
first demo should be provided, and all important decisions about technical and
logical aspect of the project should be made.

\subsubsection{Duration}
Start of Sprint 0 is September 4th, and it will last
until September 11th (week 36). During this sprint, the field study is
scheduled for September 5th. The scheduled customer meeting is on the same
day, and the advisor meeting is on September 6th.

\subsection{User Interface}
\subsubsection{Overview}
The choice of Phonegap for developing a cross compiling mobile application gave us the possibility to use HTML/CSS for application design. The team was able to use existing libraries and templates (CSS, javascript, html, etc.).
The JavaScript library, jQuery along with jQuery Mobile, with its touch-optimized layouts and its good design and simplicity was found to be a suitable choice for the application development and was therefore chosen.

\subsubsection{Choice of Template}
Artsdatabanken does have a webpage optimized for mobile phones, which also uses jQuery mobile. The customer wanted the application to have the same template as the webpage. To do that the team used the CSS file from the webpage in the application. This way, the application is managed to look very similar to the mobile webpage.
\begin{figure}[h!]
 \begin{center}$
 \begin{array}{cc}
 \includegraphics[width=0.5\textwidth,height=0.7\textwidth]{sprints/ui/main.jpg} &
 \includegraphics[width=0.5\textwidth,height=0.7\textwidth]{sprints/ui/webpage.jpg}
 \end{array}$
 \end{center}
 \caption{Similarity of the application (left) and  the webpage (right). }
 \end{figure}





\newpage
\subsubsection{Preview of the UI with Screenshots}

\begin{figure}[h]
 \begin{center}$
 \begin{array}{cc}
 \includegraphics[width=0.5\textwidth]{sprints/ui/main.jpg} &
 \includegraphics[width=0.5\textwidth]{sprints/ui/mainHover.jpg}
 \end{array}$
 \end{center}
 \caption{Main Screen, Right:Button Clicked }
 \end{figure}

\subparagraph{Main Screen}
The figure above shows the main screen of the application. The buttons are clear and the user can simply do a selection.
When the user makes a selection, the selected button becomes blue and the user is sure that the right button is pressed.

\newpage
\begin{figure}[h]
 \begin{center}$
 \begin{array}{cc}
 \includegraphics[width=0.5\textwidth]{sprints/ui/obs.jpg} &
 \includegraphics[width=0.5\textwidth]{sprints/ui/obsHover.jpg}
 \end{array}$
 \end{center}
 \caption{New Observation, Right:Button Clicked }
\label{fig:observation}
 \end{figure}

\subparagraph{New Observation}
After selecting New Observation, the page transitions into the window displayed in figure~\ref{fig:observation}. Here the user selects the Species.
Their names are displayed, with pictures on the side. More species are listed; the user needs to slide the interface using a finger to see those.

The selection will give a feedback to the user, like in the main screen.

There is a back button on the top left. This button will lead to the previous window.


\newpage
\begin{figure}[h]
 \begin{center}$
 \begin{array}{cc}
 \includegraphics[width=0.5\textwidth]{sprints/ui/bird.jpg} &
 \includegraphics[width=0.5\textwidth]{sprints/ui/autocomp.jpg}
 \end{array}$
 \end{center}
 \caption{Bird Observation, Right:Auto-Complete field }
\label{fig:bird}
 \end{figure}

\subparagraph{Bird Observation}
After selecting the species; bird, the user is given a new window (figure~\ref{fig:bird}).
This is the window where the information about the observation is added.

The user can fill the fields easily by selecting the field that are required to be to fill. The selected field is marked with a blue shadow so the user knows which field is selected.
The species field has an auto-complete function, which makes it simpler for the user to type the correct name.

\newpage
\begin{figure}[h]
 \begin{center}$
 \begin{array}{cc}
 \includegraphics[width=0.5\textwidth]{sprints/ui/add_more.jpg} &
 \includegraphics[width=0.5\textwidth]{sprints/ui/add_more2.jpg}
 \end{array}$
 \end{center}
 \caption{Add More Information, Right:New window after select }
\label{fig:addmore}
 \end{figure}
\subparagraph{Add More information}
There are several buttons here for the user. If the user wants to add more information about the species, the “Add More Information” button is selected.
This leads to a new window (shown in figure~\ref{fig:addmore}). To go back after entering new data, the user selects the back button.

\newpage
\begin{figure}[h]
 \begin{center}$
 \begin{array}{cc}
 \includegraphics[width=0.5\textwidth]{sprints/ui/add_specie.jpg} &
 \includegraphics[width=0.5\textwidth]{sprints/ui/add_specie2.jpg}
 \label{fig:addanotherone}
 \end{array}$
 \end{center}
 \caption{Add Another Species, Right:Window with the new species line }
 \end{figure}
 \subparagraph{Add New Species}
If the user wants to add new species, the user selects the “Add another species” button. This gives a new row in the same window, by adding new species the window increases its height. The user needs to slide vertically if many more species are added.


\subsection{Customer Feedback}
At the end of this sprint we held a meeting with
the customer on September 13th. The team presented the conclusions obtained in
the preliminary study, in addition to the first demo of the application. The
customer agreed upon our choices and application design, and also gave some
useful suggestions about the colors and design guidelines for Artsdatabanken.
PhoneGap proved to be a good choice for development.  As requested in the
customer meeting, the next sprint should focus on auto-completion of species
names when making a new observation.

\subsection{Evaluation}
This shorter, but not less important sprint, provided
very valuable information about the problem domain. A lot of decisions were made
that will continue to guide the rest of the development process. From the
preliminary study, the team now has a good understanding of the problem domain.
A lot of important issues were discussed in detail, such as development
environment, programming and scripting languages, field study etc. All members
were involved in this discussions, including both the advisor and the customer.
In addition, the sprint included some programming, and laid the foundation for
further development.  The customer agreed with most of our suggestions, and gave
us guidelines for further development and studies.  Documentation was done by
the whole team, and the progress was very satisfactory. However, documenting a
lot of things in a short time led to lower document quality. After consultation
with the advisor, it was decided that a lot of key changes should be made in
next the sprint, and the document should be reorganized.
