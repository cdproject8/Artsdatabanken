\section{Sprint 2}

\subsection{Sprint Planning}

	\subsubsection{Expected results}
  From this sprint we expect to have a working local storage to the phone working. This will provide persistance of observations in the phone. In addition, the document will be ready for the pre-delivery 6th of October.
	
	\subsubsection{Duration}
	Start of Sprint 2 is September 26th, and it will last until October 9th (weeks 39 and 40). Customer meeting is on October 11th, just after the completion of whole sprint. Advisor meetings are on September 27th and October 4th.

\subsection{Requirements}

\begin{description}
	\item[F10] Autocomplete should be able to load data based on species category
	\item[F2] User must be able to add more information to a species observation
	\item[F6] User must be able to view observations stored on the device
	\item[F7] User must be able to edit a stored observation on the device
\end{description}

\subsection{Implementation}

A species in the observation can now have more information added to it via the 'Extended Information' page.
A click on the details button is caught by JQuery, the id of the species is found by finding the parents of the button in the DOM, where a div contains the id.
The activeExtended attribute in Observation is set to that species and the application transitions to the Extended Information page, values from the species object is read and filled in with JQuery. 
When transitioning back to the Observation page all informatin is saved to the object again and the row of the edited species is updated on the Observation page.

The user may now view stored observations, a button for this has been made functional on the main page.
\paragraph{ObservationList.js} populates a list of saved observations.
The user can select an observation, identified by which species category it is related to and when it was created as well as it's id.
When an observation is selected, two global variables related to the id of the observation and it's speciesgroup are set and the user is redirected to the Observation page.
The observation is read from storage and loaded into the DOM, editable in the same way as when first created in order to create a simple and easily recongnizable interface.

This uses the same code as when a new observation is created, however when a new one is created the global variables are unset and a new observation object is created instead by main.js.

	\subsubsection{Auto-complete}

	Auto-complete has been extended to read files based on species category.
	Initially each category was included in a separate file, but this caused some
	performance issues due to the large amount of data (browser freeze while reading
	files). To circumvent this issue, team decided to split the categories into
	multiple files, keeping one folder per category and one file per prefix see.

	To achieve this, following conventions / specifications will be used:

	\begin{itemize}
		\item Auto-Complete data is rooted in the directory WEBROOT/data/autocomplete
		\item Each species category will get a sub-folder under the auto-complete
		root. This will be named after the category id (an integer).
		\item Each category directory should contain one index.js file that
		keeps an overview of all the prefix-files for the category.
		\item Files in the category directory are named by the following rules:
			\begin{itemize}
				\item a.json contains only entries prefixed by a
				\item a\_b.json contains entries prefixed by a or b (can be
				extended with more entries)
				\item a-z.json contains entries prefixed by a, b, c, ..., or z
				\item a\_c-e\_q\_t contains entries prefixed by a, c, d, e, q, or t
			\end{itemize}
	\end{itemize}

\subsection{Testing}

Auto-complete functionality was extended using test driven development, unit
tests were written before functionality was implemented (red-green).

\subsection{Customer Feedback}

\subsection{Evaluation}
