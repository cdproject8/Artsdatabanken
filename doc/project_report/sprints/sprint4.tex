\section{Sprint 4}

\subsection{Sprint Planning}

This is the last sprint in this project. The group intended to finish all the
requirements in the sprint backlog and formally close major functionality
development. 
In this sprint, the teams focus is on rectifying previous sprints technical
errors, and completing the export functionality of the application. We will also
look at GPS and camera functionalities.

\subsubsection{Expected results}
Sprint 4 is the last of our formal sprints, and all requirements will be
completed by the end of this sprint. The expected result is a working export
functionality, in addition to GPS coordinates and working camera functionality.

\subsubsection{Duration}
Sprint 4 began on October 24th in week 43 and will last on November 6th in week
44. Sprint 4 has a duration of two weeks just like any of the other sprints.
The team held its customer meeting for this sprint on Tuesday October 25, 2011.
The advisor meeting, scheduled for October 25th was cancelled due to absence. 

\subsection{Requirements}
Some of the requirements are from sprint 2 and the goal is to make them free
from bugs and issues and focus more on exporting the observation.

\begin{description}
	\item[F5] User must be able to take a picture with the devices camera.
	\item[F8] User must be able to update the local database of species and locations
	\item[F9] An observation must contain GPS coordinates of where it was created.
\end{description}

\subsection{Implementation}
An observation can now be updated with GPS coordinates, which is generated by
native geolocation features presented by the PhoneGap API and works on all
supported PhoneGap platforms.

The activity box has been filled with the provided list of valid activities, but
because there are so many, we implemented a select plugin called Chosen
\cite{library:chosen}. This is a hybrid of select box and autocomplete, but in
order for this to function properly within the jQuery Mobile framework we had to
create an invisible input field, append the html code for chosen and activate it
after the page has loaded. We did it this way in order to avoid jQuery Mobiles
restructuring of GUI elements when creating a page.

In order to allow the user to attach pictures to the observations, two buttons
were added to the details page of the observation. One opens the default camera
app of the device, the other opens the default album app. Each returns an uri to
the image selected/taken, which is stored in the database along with the
observation/species IDs (see section \ref{sec:storage} for details). The
pictures attached to a species in an observation are shown on the details page.
Pictures can be removed from a species by tapping them and confirming. However
the pictures themselves are not deleted from the device.

The export functionality was updated to include pictures, the Java plugin was
extended to be able to recieve URIs to pictures on the device.  These pictures
are sent along with the observation string so that the email client will include
these pictures.

There was a small issue with including several pictures and not just
one.  It seems this is not supported by all email clients on Android and may
cause issues if exporting an observation with pictures if the phone does not
possess such an application.  This functionality has been tested and confirmed
to work with the GMail and SonyEricsson email clients.

	\subsubsection{Auto-complete}

  Implemented tools for downloading and parsing data from the new API provided
  by Artsdatabanken. This required us to change some URL's and use literal names
  instead of ID's in requests. Using literal names was a bit problematic as
  Artsdatabanken is including some special characters in the literal names that
  are not actually supported by the API (using commas in a request causes the API
  to return an SQL error).

  We ran into some problems with the new API however. We could only retrieve 500
  species at a time. The logic behind the added requests and the storing of data
  on the phone as files is very complex, making it a project in itself. We
  therefore chose a different approach by making the application have periodic
  updates via the Android Market. We have created a script that can parse the
  API and create the appropriate JSON-files for use in the application. This
  will also be included in the finished product. 

\subsection{Testing}

Testing uncovered inconsistencies in the storage API. Initially we planned to
use a function , Database.changeVersion(..), to migrate database schemas while
updating the app, unfortunately this only works for some versions of Android.
Due to preceding comment and time restrictions we dropped support for database
migration.

We ran functional tests for each of the completed requirements (details in
section \ref{sec:testing}). We tested each use case on different devices, after
each test bug fixing was performed. Some of the improvements can be seen in run
1 and run 3, run 3 is a re-test after fixing bugs in version 0.9 (done on the
same device). 

At the end all tests passed. Test 5 uncovered a weakness in our user interface,
we have a button named "Ta bilde" (capture image) that is used to choose images
from the device album, it doesn't allow users to actually capture an image.
Image capturing is actually done from another section in the user interface
(details in section \ref{sec:testing}. To accommodate this we created an
additional test, "Test 1 Fixed", that verifies the actual image capturing
functionality.

\subsection{Customer Feedback}
We received positive feedback from the customer about our final prototype. While
we felt the application was not ready for production just yet, we had laid the
foundation and groundwork for a good application. In addition, the customer
applauded our research into cross-compiling frameworks.

The customer felt the project was a success, and instructed us to continue work
on our documentation for the remaining period. 

\subsection{Evaluation}
At the end of sprint 4 we had completed all of the functional requirements and
exhausted our product backlog. The project was deemed a success by the customer
himself, and we felt that with the prototype now complete, we could focus on the
remaining documentation.

For the work in sprint itself, we did not have many issues. Our biggest gripe
was with multiple file attachments in certain cellphones' email client. Some did
simply not have the functionality to include several attachments. No workaround
was found, but it worked for the standard GMail-client installed on every phone. 
