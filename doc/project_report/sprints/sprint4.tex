\section{Sprint 4 TODO}

	\subsection{Sprint Planning}
This is the last sprint in this project. The group intended to finish all the requirements in the sprint backlog and formally close major functionality development. In this sprint, the team worked on rectifying the technical errors in the implementation of early sprint requirements, especially getting to work the exporting functionality fully. GPS location and camera functionalities are also planned to be completed here.
	\subsubsection{Expected results}
Sprint 4 will mark the end of our sprints and all the requirements will have been completed and working by the end of this sprint. The expected result is the export functionality, the GPS location and Picture taking (camera) functionality will be working.	
	\subsubsection{Duration}
Sprint 4 began on October 24th in week 43 and will last on November 6th in week 44. Sprint 4 has a duration of two weeks just like any of the other sprints. The team held its customer meeting for this sprint on Tuesday October 25, 2011 and the advisor meeting was on the same date in the afternoon.
	\subsection{Requirements}
Some of the requirements are from sprint 2 and the goal is to make them free from bugs and issues and focus more on exporting the the observation.
\begin{description}

\item[F5] User must be able to take a picture with the devices's camera.

\item[F8] User must be able to update the local database of species and locations

\item[F9] An observation must contain GPS coordinates of where it was created.

\end{description}
	\subsection{Implementation}
	\subsection{Testing}

	\subsection{Customer Feedback}

	\subsection{Evaluation} 
F8 was the most challenging requirement other than the exporting functionality. We had to roll on exporting functionality across 3 sprint, but the team managed to complete that in time... 
