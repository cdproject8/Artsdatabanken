\section{Sprint 4 TODO}

\subsection{Sprint Planning}

This is the last sprint in this project. The group intended to finish all the
requirements in the sprint backlog and formally close major functionality
development. In this sprint, the team worked on rectifying the technical errors
in the implementation of early sprint requirements, especially getting to work
the exporting functionality fully. GPS location and camera functionalities are
also planned to be completed here.

\subsubsection{Expected results}

Sprint 4 will mark the end of our sprints and all the requirements will have
been completed and working by the end of this sprint. The expected result is
the export functionality, the GPS location and Picture taking (camera)
functionality will be working.	

\subsubsection{Duration}

Sprint 4 began on October 24th in week 43 and will last on November 6th in week
44. Sprint 4 has a duration of two weeks just like any of the other sprints.
The team held its customer meeting for this sprint on Tuesday October 25, 2011
and the advisor meeting was on the same date in the afternoon.

\subsection{Requirements}

Some of the requirements are from sprint 2 and the goal is to make them free
from bugs and issues and focus more on exporting the the observation.

\begin{description}
	\item[F5] User must be able to take a picture with the devices's camera.
	\item[F8] User must be able to update the local database of species and locations
	\item[F9] An observation must contain GPS coordinates of where it was created.
\end{description}

\subsection{Implementation}
An observation can now be updated with GPS coordinates, these are generated by native geolocation features presented by the PhoneGap API and works on all supported PhoneGap platforms.

TODO documentation about picuteres

The export functionality was updated to include pictures, the Java plugin was extended to be able to recieve URIs to pictures on the device.
These pictures are send along with the observation string so that the email client will include these pictures.

There was however a small issue with including several pictures and not just one.
It seems this is not supported by all email clients on android and may cause issues if exporting an observation with pictures if the phone does not possess such an application.
This functionality has been tested and confirmed to work with the GMail and SonyEricsson email clients.

	\subsubsection{Auto-complete}

	Implemented tools for downloading and parsing data from the new API
	provided by Artsdatabanken. This required us to change some URL's and use
	literal names instead of ID's in requests. Using literal names was a bit
	problematic as Artsdatabanken is including some special characters in the
	literal names that are not actually supported by the API (using commas in a
	request cause the API to return an SQL error).

  We ran into some problems with the new API however. We could only retrieve 500 species at a time. The logic behind the added requests and the storing of data on the phone as files is very complex, making it a project in itself. We therefore chose a different approach by making the application have periodic updates via the Android Market. We have created a script that can parse the API and create the appropriate JSON-files for use in the application. This will also be included in the finished product. 

\input{sprints/testing/run1.tex}
\newpage
\subsection{Test 1 (run 3)}

	\begin{figure}[htb]
		\centering
		\begin{tabular}{|p{3.5cm}|p{7.0cm}|} \hline
			\textbf{Requirements} & F1 and F10 \\ \hline
			\textbf{Version} & 1.0 \\ \hline
			\textbf{Date} & 2011-11-08 \\ \hline
			\textbf{Tested by} & Stian Liknes \\ \hline
			\textbf{Test environment} & Sony Ericsson Xperia X10 running Android 2.1.1.A.0.6 with kernel: 2.6.29 \\ \hline
			\textbf{Pre-conditions} & Clean install of app on mobile device, no observations stored \\ \hline
			\textbf{Post-conditions} & A new observation has been saved and the user is directed back to the main menu \\ \hline
			\textbf{Result} & PASS \\ \hline
		\end{tabular}
		\caption{Summary of test 1}
	\end{figure}

	\begin{figure}[htb]
		\centering
		\begin{tabular}{|p{5.0cm}|p{5.0cm}|p{1cm}|}
			\hline \textbf{Action} & \textbf{Expected outcome} & \textbf{Result} \\ \hline
			1. Tap the new observation button & Menu for selecting species group appears & PASS \\ \hline
			2. Select species type "Fugl" (bird) & Menu for bird observations appears & PASS \\ \hline
			3. Selects location from list of close locations, or selects GPS location & Location is set & PASS \\ \hline
			4. Start writing "grågås" in the "Art" (species) field, ensure that
			auto-complete give useful suggestions, choose "grågås" from list of
			suggestions. Write 2 in the "Antall" (count) field & Auto-complete
			suggests bird names, 2 species of type "grågås" is added to observation
			& PASS \\ \hline 
			5. Tap "Lagre". & Observatin is saved & PASS \\ \hline
		\end{tabular}
		\caption{Execution of test 1}
	\end{figure}

\newpage
\subsection{Test 2 (run 3)}

	\begin{figure}[htb]
		\centering
		\begin{tabular}{|p{3.5cm}|p{7.0cm}|} \hline
			\textbf{Requirement} & F2 \\ \hline
			\textbf{Version} & 1.0 \\ \hline
			\textbf{Date} & 2011-11-08 \\ \hline
			\textbf{Tested by} & Stian Liknes \\ \hline
			\textbf{Test environment} & Sony Ericsson Xperia X10 running Android 2.1.1.A.0.6 with kernel: 2.6.29 \\ \hline
			\textbf{Pre-conditions} & Test 1 completed in same test environment, app is still in observation view \\ \hline
			\textbf{Post-conditions} & Additional information about an observation has been saved \\ \hline
			\textbf{Result} & PASS \\ \hline
		\end{tabular}
		\caption{Summary of test 2}
	\end{figure}

	\begin{figure}[htb]
		\centering

		\begin{tabular}{|p{5.0cm}|p{5.0cm}|p{1cm}|}
			\hline \textbf{Action} & \textbf{Expected outcome} & \textbf{Result} \\ \hline

			1. Tap "Detaljer" (details) in the row containing "grågås" &
			Detailed view for the "grågås" observation is displayed & 
			PASS \\ \hline

			2. Select "Rugende" in the "Aktivitet" (activity) field &
			"Aktivitet" field populated with "Rugende" &
			PASS \\ \hline

			3. Tap back button & 
			Main observation view is displayed & 
			PASS \\ \hline

			4. Tap save and go to main screen. Go into the observatin using
			"Lagrede Observasjoner" (stored observations) and verify that "grågås"
			still has "Aktivitet" set to "Rugende" in the details view &
			Grågås has "Aktivitet" set to "Rugende" &
			PASS \\ \hline
		\end{tabular}
		\caption{Execution of test 2}
	\end{figure}

\newpage
\subsection{Test 3 (run 3)}

	\begin{figure}[htb]
		\centering
		\begin{tabular}{|p{3.5cm}|p{7.0cm}|} \hline
			\textbf{Requirement} & F3 \\ \hline
			\textbf{Version} & 1.0 \\ \hline
			\textbf{Date} & 2011-11-08 \\ \hline
			\textbf{Tested by} & Stian Liknes \\ \hline
			\textbf{Test environment} & Sony Ericsson Xperia X10 running Android 2.1.1.A.0.6 with kernel: 2.6.29 \\ \hline
			\textbf{Pre-conditions} & Test 1 completed in same test environment, app is still in observation view \\ \hline
			\textbf{Post-conditions} & Observation is stored with two entries, "grågås" ("Antall" of 2) and "blåmeis" ("Antall" of 1) \\ \hline
			\textbf{Result} & PASS \\ \hline
			\textbf{Comments} & We decided to remove step 2 from the use case \\ \hline
		\end{tabular}
		\caption{Summary of test 3}
	\end{figure}

	\begin{figure}[htb]
		\centering
		\begin{tabular}{|p{5.0cm}|p{5.0cm}|p{1cm}|}
			\hline \textbf{Action} & \textbf{Expected outcome} & \textbf{Result} \\ \hline

			1. Tap "Legg til ny art" &
			A empty species row is appended &
			PASS \\ \hline
			
			2. Optionally selects another location, otherwise the same one is
			selected. & 
			New location chosen for current row &
			- \\ \hline

			3. Select "blåmeis" with the count ("Antall") of 1 in the same matter 
			as in test 1. &
			Row 2 is filled inn with ("Artsnavn" = "blåmes", "Antall" = 1) &
			PASS \\ \hline

			4. Tap "Lagre" &
			Observation is saved with two species observations ("grågås" and "blåmeis") &
			PASS \\ \hline
		\end{tabular}
		\caption{Execution of test 3}
	\end{figure}

\newpage
\subsection{Test 4 (run 3)}

	\begin{figure}[htb]
		\centering
		\begin{tabular}{|p{3.5cm}|p{7.0cm}|} \hline
			\textbf{Requirement} & F4 \\ \hline
			\textbf{Version} & 1.0 \\ \hline
			\textbf{Date} & 2011-11-08 \\ \hline
			\textbf{Tested by} & Stian Liknes \\ \hline
			\textbf{Test environment} & Sony Ericsson Xperia X10 running Android 2.1.1.A.0.6 with kernel: 2.6.29 \\ \hline
			\textbf{Pre-conditions} & Test 2 and 3 completed in same test environment, app is still in observation view \\ \hline
			\textbf{Post-conditions} & All data from current observation is submitted to the native mail client \\ \hline
			\textbf{Result} & PASS \\ \hline
		\end{tabular}
		\caption{Summary of test 4}
	\end{figure}

	\begin{figure}[htb]
		\centering
		\begin{tabular}{|p{5.0cm}|p{5.0cm}|p{1cm}|}
			\hline \textbf{Action} & \textbf{Expected outcome} & \textbf{Result} \\ \hline
			1. Taps 'Eksporter' (export) in the observation view &
			Native email client is launched in "new email"-mode. Date from
			observation is placed in the message field. &
			PASS \\ \hline
		\end{tabular}
		\caption{Execution of test 4}
	\end{figure}

\newpage
\subsection{Test 5 (run 3)}

	\begin{figure}[htb]
		\centering
		\begin{tabular}{|p{3.5cm}|p{7.0cm}|} \hline
			\textbf{Requirement} & F5 \\ \hline
			\textbf{Version} & 1.0 \\ \hline
			\textbf{Date} & 2011-11-08 \\ \hline
			\textbf{Tested by} & Stian Liknes \\ \hline
			\textbf{Test environment} & Sony Ericsson Xperia X10 running Android 2.1.1.A.0.6 with kernel: 2.6.29 \\ \hline
			\textbf{Pre-conditions} & App installed on mobile device \\ \hline
			\textbf{Post-conditions} & Picture is stored on the phone with an easily recognizable filename \\ \hline
			\textbf{Result} & FAIL \\ \hline
			\textbf{Comments} & Could only take picture when no other pictures existed on device. Othewise you can select existing images.\\ \hline
		\end{tabular}
		\caption{Summary of test 5}
	\end{figure}

	\begin{figure}[htb]
		\centering
		\begin{tabular}{|p{5.0cm}|p{5.0cm}|p{1cm}|}
			\hline \textbf{Action} & \textbf{Expected outcome} & \textbf{Result} \\ \hline
			1. Tap "Ta Bilde" (capture image) in the main view &
			Native image capturing software is started & 
			PASS \\ \hline

			2. Take picture using native software &
			Image is stored and success message is displayed in app &
			FAIL \\ \hline
		\end{tabular}
		\caption{Execution of test 5}
	\end{figure}

\newpage
\subsection{Test 6 (run 3)}

	\begin{figure}[htb]
		\centering
		\begin{tabular}{|p{3.5cm}|p{7.0cm}|} \hline
			\textbf{Requirement} & F6 \\ \hline
			\textbf{Version} & 1.0 \\ \hline
			\textbf{Date} & 2011-11-08 \\ \hline
			\textbf{Tested by} & Stian Liknes \\ \hline
			\textbf{Test environment} & Sony Ericsson Xperia X10 running Android 2.1.1.A.0.6 with kernel: 2.6.29 \\ \hline
			\textbf{Pre-conditions} & Test 2 completed \\ \hline
			\textbf{Post-conditions} & App is in same state as before test started \\ \hline
			\textbf{Result} & PASS \\ \hline
		\end{tabular}
		\caption{Summary of test 6}
	\end{figure}

	\begin{figure}[htb]
		\centering
		\begin{tabular}{|p{5.0cm}|p{5.0cm}|p{1cm}|}
			\hline \textbf{Action} & \textbf{Expected outcome} & \textbf{Result} \\ \hline
				1. Tap "Lagrede Observasjoner" from the main view &
				A list containing one observatin (from test 2) is displayed &
				PASS \\ \hline

				2. Select observation 1 from the list &
				The observation from test 2 is displayed in the same state as
				earlier (two species) &
				PASS \\ \hline

				3. Tap "Detailer" in the row containing "grågås" &
				Details view for "grågås" observation is displayed, the field "Aktivitet"
				is filled in with "Rugende" &
				PASS \\ \hline
		\end{tabular}
		\caption{Execution of test 6}
	\end{figure}

\newpage
\subsection{Test 7 (run 3)}

	\begin{figure}[htb]
		\centering
		\begin{tabular}{|p{3.5cm}|p{7.0cm}|} \hline
			\textbf{Requirement} & F7 \\ \hline
			\textbf{Version} & 1.0 \\ \hline
			\textbf{Date} & 2011-11-08 \\ \hline
			\textbf{Tested by} & Stian Liknes \\ \hline
			\textbf{Test environment} & Sony Ericsson Xperia X10 running Android 2.1.1.A.0.6 with kernel: 2.6.29 \\ \hline
			\textbf{Pre-conditions} & Test 6 completed, still viewing stored observation \\ \hline
			\textbf{Post-conditions} & Observation is stored on device with an additional row containing ("Art" = "grønnfink", "Antall" = 9) \\ \hline
			\textbf{Result} & PASS \\ \hline
		\end{tabular}
		\caption{Summary of test 7}
	\end{figure}

	\begin{figure}[htb]
		\centering
		\begin{tabular}{|p{5.0cm}|p{5.0cm}|p{1cm}|}
			\hline \textbf{Action} & \textbf{Expected outcome} & \textbf{Result} \\ \hline

			1. Add a new row with the same procedure as in test 2 &
			A empty row is appended to the observation &
			PASS \\ \hline

			2. Fill in ("Art" = "grønnfink" and "Antall" = 9) in the new row
			using the same procedure as in test 2. &
			New row is populated with ("grønnfink", 9) &
			PASS \\ \hline

			3. Tap "Lagre" &
			Observation stored with an additional row ("grønnfink", 9) &
			PASS \\ \hline

		\end{tabular}
		\caption{Execution of test 7}
	\end{figure}

\newpage
\subsection{Test 8 (run 3)}

	\begin{figure}[htb]
		\centering
		\begin{tabular}{|p{3.5cm}|p{7.0cm}|} \hline
			\textbf{Requirement} & F9 \\ \hline
			\textbf{Version} & 1.0 \\ \hline
			\textbf{Date} & 2011-11-08 \\ \hline
			\textbf{Tested by} & Stian Liknes \\ \hline
			\textbf{Test environment} & Sony Ericsson Xperia X10 running Android 2.1.1.A.0.6 with kernel: 2.6.29 \\ \hline
			\textbf{Pre-conditions} & Test 1 completed, still viewing stored observation \\ \hline
			\textbf{Post-conditions} & App is in same state as before test execution \\ \hline
			\textbf{Result} & PASS \\ \hline
		\end{tabular}
		\caption{Summary of test 8}
	\end{figure}

	\begin{figure}[htb]
		\centering
		\begin{tabular}{|p{5.0cm}|p{5.0cm}|p{1cm}|}
			\hline \textbf{Action} & \textbf{Expected outcome} & \textbf{Result} \\ \hline
			
			1. Tap "GPS" &
			Longitude and latitude contains coordinates near the current location &
			PASS \\ \hline

		\end{tabular}
		\caption{Execution of test 8}
	\end{figure}

\newpage



\subsection{Customer Feedback}

\subsection{Evaluation}

F8 was the most challenging requirement other than the exporting functionality.
We had to roll on exporting functionality across 3 sprint, but the team managed
to complete that in time...
