\section{Sprint 4 TODO}

\subsection{Sprint Planning}

This is the last sprint in this project. The group intended to finish all the
requirements in the sprint backlog and formally close major functionality
development. In this sprint, the team worked on rectifying the technical errors
in the implementation of early sprint requirements, especially getting to work
the exporting functionality fully. GPS location and camera functionalities are
also planned to be completed here.

\subsubsection{Expected results}

Sprint 4 will mark the end of our sprints and all the requirements will have
been completed and working by the end of this sprint. The expected result is
the export functionality, the GPS location and Picture taking (camera)
functionality will be working.	

\subsubsection{Duration}

Sprint 4 began on October 24th in week 43 and will last on November 6th in week
44. Sprint 4 has a duration of two weeks just like any of the other sprints.
The team held its customer meeting for this sprint on Tuesday October 25, 2011
and the advisor meeting was on the same date in the afternoon.

\subsection{Requirements}

Some of the requirements are from sprint 2 and the goal is to make them free
from bugs and issues and focus more on exporting the the observation.

\begin{description}
	\item[F5] User must be able to take a picture with the devices's camera.
	\item[F8] User must be able to update the local database of species and locations
	\item[F9] An observation must contain GPS coordinates of where it was created.
\end{description}

\subsection{Implementation}
An observation can now be updated with GPS coordinates, these are generated by native geolocation features presented by the PhoneGap API and works on all supported PhoneGap platforms.

TODO documentation about picuteres

The export functionality was updated to include pictures, the Java plugin was extended to be able to recieve URIs to pictures on the device.
These pictures are send along with the observation string so that the email client will include these pictures.

There was however a small issue with including several pictures and not just one.
It seems this is not supported by all email clients on android and may cause issues if exporting an observation with pictures if the phone does not possess such an application.
This functionality has been tested and confirmed to work with the GMail and SonyEricsson email clients.

	\subsubsection{Auto-complete}

	Implemented tools for downloading and parsing data from the new API
	provided by Artsdatabanken. This required us to change some URL's and use
	literal names instead of ID's in requests. Using literal names was a bit
	problematic as Artsdatabanken is including some special characters in the
	literal names that are not actually supported by the API (using commas in a
	request cause the API to return an SQL error).

  We ran into some problems with the new API however. We could only retrieve 500 species at a time. The logic behind the added requests and the storing of data on the phone as files is very complex, making it a project in itself. We therefore chose a different approach by making the application have periodic updates via the Android Market. We have created a script that can parse the API and create the appropriate JSON-files for use in the application. This will also be included in the finished product. 

\subsection{Testing}

Testing uncovered inconsistencies in the storage API. Initially we planned to
use a function , Database.changeVersion(..), to migrate database schemas while
updating the app, unfortunately this only works for some versions of Android.
Due to preceding comment and time restrictions we dropped support for database
migration.

We ran functional tests for each of the completed requirements (details under
\ref{sec:testing}). We tested each use case on different devices, after each test bug
fixing was done. The improvements can be seen in run 1 and run 3, run 3 is a
re-test after fixing bugs in version 0.9 (done on the same device). At the end
all tests passed. Test 5 uncovered a weakness in our user interface, we have a
button named "Ta bilde" (capture image) that is used to choose images from the device
album, it doesn't allow users to actually capture an image.

\subsection{Customer Feedback}

\subsection{Evaluation}

F8 was the most challenging requirement other than the exporting functionality.
We had to roll on exporting functionality across 3 sprint, but the team managed
to complete that in time...
