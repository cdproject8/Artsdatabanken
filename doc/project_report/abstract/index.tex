\section{Abstract}

Artsdatabanken is a organization under the Department of Education that is
responsible for species naming and observations. They’ve currently got an online
system for registering such observations and are now looking for an interface
usable on mobile devices.

By observing users we’ve collected knowledge of the process of such
observations, how they are conducted and how they are recorded. An observation
starts off with a person identifying species at a location and noting them down,
then additional details are recorded as well if desirable. Usually in a
notebook, and the later on entered manually on a website. We intend to replace
this notebook with a mobile application to make this process easier and faster,
and also to make it more approachable for new users of the services of
Artsdatabanken.

We developed the application using the cross-compiling framework Phonegap. This
framework utilizes HTML5, CSS3 and JavaScript to develop applications that live
in the browser of the phone. The application itself helps the users create
observations and exporting them for easy upload to Artsdatabankens service. 
