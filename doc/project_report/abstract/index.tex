\section{Abstract}

Motivation, problem statement, approach, results and conclusions

\paragraph{Motivation}
Artsdatabanken, an organization under the Norwegian Department of Education has
long needed a mobile application for species observation. Observers in the field
still use a paper notebook to do observations, and are requesting a simpler and
more effective means to log and register their findings with Artsdatabanken.

\paragraph{Problem statement}
Our task was to develop a cross-platform mobile application that could simplify
registration of species, with automatic gathering of information such as GPS and
pictures, and provide a means to easily register this with Artsdatabankens
systems. The system should replace the old paper notebook, work on mobile
devices such as the iPhone, iPad and Android devices, and reduce the time needed
to register your observations online.

In addition to this, we were tasked with performing research into different
cross-compiling frameworks, and draw a conclusion about which framework worked
the best. We would also evaluate this framework throughout our process to see if
using such a framework was viable as a means to develop cross-platform
applications.

\paragraph{Approach}
Our approach consisted of performing a thorough study of frameworks and the
method of species observations. Amongst other things, we performed a field study
at Artsdatabanken about species observation, and we held workshops discussing
the features of the application. 

The application was developed using the PhoneGap framework, and we focused
primarily on the Android platform. We focused a lot on testing, and used
Test-Driven Development, in addition to performing a usability test of the
application.

The group also established industry-proven quality assurance, and used an agile
methodology based on SCRUM for the project. 

\paragraph{Results}
The final application should work on all mobile devices, but remain untested for
other platforms than Android because of the lack of hardware for testing. The
application is programmed using PhoneGap, in combination with jQuery Mobile as the
interface framework.

Our usability testing revealed that our app has some good points, but need some
improvements before it can be put into production. It seems to have a user
friendly interface, but the installation process is not satisfactory. On
average, testers are positive to use the app and think that it can help them be
more effective in species observations. The majority disagrees in that the app
is smooth and flawless. These results are based on a test run on only three users,
therefore the conclusions should be taken with a grain of salt.

\paragraph{Conclusion}
TODO


