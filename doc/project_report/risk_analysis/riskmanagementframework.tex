%todo review references..
\subsection{Risk Management Framework}

Risk is an inherent part of any project. Risk can come in several forms such as
projects violating budget or schedule, losing track of goal, missing the
essence of the task on hand, etc. In our analysis, we will consider project
management , technical difficulties, shortage of man power, unplanned events
and application security .\cite{wiki:rmf1}

There are four key tasks of risk management planning\cite{wiki:rmf2}
\begin{enumerate}
    \item{Identify risks}
	\item{Quantify risks}
	\item{Develop counter-measures}
    \item{Regularily review risk analysis}
\end{enumerate}

Risks will be quantified by likelihood and impact, this can later be used for
prioritizing. We use definitions listed in table \ref{tab:risk-likelihood} and
\ref{tab:risk-impact} in the quantification step.

\begin{center}
    \begin{table}[htb]
    \begin{tabular}{ | l | p{10cm} |}

    \hline
    Title & Description  \\ \hline \hline
    Very Low & Highly unlikely to occur; however, still needs to be monitored as certain circumstances could result
    in this risk becoming more likely to occur during the project. \\ \hline
    Low & Based on current information unlikely to occur, and the circumstances likely to trigger the risk are also unlikely to occur. \\ \hline
    Medium & Likely to occur. Some indicators that the risk might probably materialize.  \\ \hline
    High & Current circumstances show that the the risk is very likely to occur. \\ \hline
    Very High & Some events indicate that it some the risk is highly likely to occur and measures might not stop it from materializing.\\ \hline
    \end{tabular}
    \caption{Risk likelihood measurement parameters}
    \label{tab:risk-likelihood}
    \end{table}
    
    \begin{table}[htb]
    \begin{tabular}{|l|p{10cm}|}
    \hline
    Title & Description \\ \hline \hline
    Very Low & Insignificant impact on the project. Minor discomfort. \\ \hline
    Low & Minor impact on the project, less significant milestones not met. \\ \hline
    Medium & Project could be delayed, a lot of work to meet deadlines but manageable with a mitigation plan. \\ \hline
    High & A major problem, significant revision in plan or product. A serious of delays in deliverables.\\ \hline
    Very High & Over budget and over schedule. Total collapse of the project. \\ \hline
    \end{tabular}
    \caption{Risk impact measurement parameters}
    \label{tab:risk-impact}
    \end{table}
\end{center}
