\section{System architecture}
\begin{figure}[htb]
	\centering
	\includegraphics[width=1.0\textwidth]{architecture/architecture.png}
	\caption{Overall system architecture}
	\label{fig:architecture}
\end{figure}

We use a layered architecture, platform-specific issues are handled using
PhoneGap as a platform-independent layer above the operating system. Following
is a detailed description of each layer, starting at the bottom.

\subsection{Layer 1 - Native platform}

This layer represents the native operating system for each device (Android, iOs,
blackberry, etc.). Services provided by this layer includes data input/output
operations, geolocation (for some devices), and so on. The operating system is
our interface to the hardware.

\subsection{Layer 2 - Cross-platform framework}

Layer 2 provides device independent abstractions for file operations, camera
access, geolocation, etc. The primary purpose of this layer is to allow us to
make portable code that can be used on Android, iOs, and other operating systems
with minimal (or no) changes to the code. PhoneGap provides an HTML5,
JavaScript, and CSS interface for the layer above.

\subsection{Layer 3 - Mobile app}

Layer 3 is the actual app, this is where our implementation will be placed.
This layer is split into an internal structure where we use the MVC pattern combined
with some layering. In addition to the components shown in the diagram, the
jQuery-family of frameworks/utilities is used by model, view and controller.

	\subsubsection{Data access}
	
	The data access sub-layer is responsible for all I/O-operations. This is
	where we access local- and remote storage. The data access layer provides
	domain centric functions for accessing common data sources. E.g.
	ObservationDAO is used to store/retrieve observations to/from local storage.

	\subsubsection{Model}

	The model will be used to represent data used in the app. We can for example
	create a class, Observation, to represent all data related to an
	observation.

	\subsubsection{Controller}

	The controller is responsible for coordinating different parts of the system,
	like handling (binding) events.

	\subsubsection{View}

	This component represents the graphical user interface, it entails code for
	generating visual effects, and updating the screen. To save work and reduce potential typing erros, we will be
	utilizing functionality (like auto-complete, and common mobile UI
	functionality) from the jQuery-family.

\subsection{Communication}

The app will be shipped with auto-complete data (primarily species names)
downloaded from Artsdatabanken's web services to allow for convenient typing in
offline mode. It will also be prepared to communicate directly with
Artsdatabanken's web services with the purpose of uploading observations and
downloading species information (for the prototype we will use a communication
format specified by Artsdatabanken, the actual communication will not be
prioritized as the API for uploading observations is not yet available).
