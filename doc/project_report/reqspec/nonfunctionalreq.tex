\newpage
\subsection{Nonfunctional requirements}
Non-functional requirements are system quality attributes or constraints that describe how a system performs its prime functionality.
They are characterized by having no clear cut criteria to determine or their criteria
changes depending on the environment and development strategies\cite{req:requirements-technique2}.
Non-functional requirements state a series of quality or constraints that could influence service delivery,
development choice, system owner and end user needs.
They are used to specify the criteria used to judge the overall operations of a system,
rather than specific behaviors.
The non-functional requirements for this project are elicited in accordance with the ISO 9126 document recommendation.
The ISO 9126 particularly focuses on quality in use of a software system.
A short summarization of ISO 9126 is described in the development process section of this documentation.
\\[0.4cm]

\subsubsection{Quality of service}
    \paragraph{Security, reliability and performance}
	\begin{itemize}
        \item The system will speed up the observation process compared to the previous best method.
        \item The system will not lose offline data.
        \item The system will prevent a breach of privacy of observer.
        \item The system will provide full mobility to the observer.

	\end{itemize}
    \paragraph{Accuracy}
        \begin{itemize}
            \item The system will store data without any normalization.
            \item The system will prevent alteration of input data during data entry processes.
            \item The system will not approximate data during data entry processes.
            \item The system will prevent alteration of picture quality through compression.
            \item The system will correctly and accurately export the data.

        \end{itemize}
    \paragraph{Interface}
        \begin{itemize}
            \item The user interface will be clean.
            \item The user interface will fit the observer device.
            \item The user interface will have legible font type/size
            \item The user interface will match the Artsdatabanken style.
            \item The user interface will have an intuitive design
            \item The user interface will be easy to use.
            \item The user interface will be fast.
            \item The user interface will operate smoothly across multiple platforms.

        \end{itemize}
\subsubsection{Compliance}
		\begin{itemize}
		     \item The system will prevent violation of device usage agreement.
             \item The system will comply with user privacy requirement.
             \item The system will comply with device platform recommendations.
             \item The system will comply with security requirements of the software owners.
	    \end{itemize}
\subsubsection{Architectural design}
    \begin{itemize}
        \item The system will be built with market tested software architecture pattern.
        \item The system will have a distributed client.
        \item The system will have easy installation.

    \end{itemize}
\subsubsection{Development paradigm}
    \begin{itemize}
        \item The system will follow a cost effective development paradigm.
        \item The system will have maintainable system components.
        \item The system will have easily replaceable components
        \item The system will be durable.
        \item The system will be adaptable to new technologies.
    \end{itemize}
