\subsection{General overview}

\subsubsection{The process}
In our first meeting with the customer we got information about what they do and what they want from us.
Both functional and non-functional requirements were discussed in general.
From this our group got an overview of our objectives.
We had some issues to take up with them about a non-functional requirement; this is described in detail later this section.
After this meeting the group members made a preliminary study to get a better overview of the task.

Our task became more clearer in the second customer meeting.
This time the meeting was on their premises, where we had a field-study and saw how the
application was to be used. Here we got more details on the functional requirements.
After the field study we had a meeting with the customer, and discussed the requirements
(functional and non-functional). We talked about functions that were most important,
and made a prioritization label for them.

\subsubsection{Project directive}

The purpose of this project is to deliver an application that will satisfy the
customer's expectations.  This application is meant to be a facilitator for the
users of Artsdatabanken, that will make the registration of an observation
simpler and more effective.  At this point in time the observer has to make
notes (in a book) while observing, and then enter the noted data on
Artsdatabanken's web page.  The application's purpose is to replace the
notebook, so users can make notes in the application.  This way the observer
doesn't need to spend time entering the observation data on the web page
manually.  This will be done by the mobile application. Saved data on the
mobile will be exported to a format that is easily submittable to the website,
whenever the mobile application gets access to an Internet connection.

To reach this goal, the application needs to be simple to use and have
sufficient functions to be preferred to a notebook.



\subsubsection{Target audience}
	
The Application is targeted at both professional users who will "use whatever they're given",
yet have different main focuses than more casual users.

The (main) target of this application will be a group that already are familiar with Artsdatabanken.
In general they will have knowledge of registering data on the web page, and will therefore not
have difficulties with the use of this application.

Also there is a potential for new users who might start using the services of Artsdatabanken if
given an easy to use interface.


\subsubsection{Project scope}
The application will be used to enter data of observations.
An observation will have sufficient data fields.
It will be possible to make multiple observations.
Stored observations in the mobile can be exported to the web page.
Observations created earlier may also be edited and re-exported.

This project's scope is to deliver the correct data to the web page.
Further processing, after that, is not covered in this project.
